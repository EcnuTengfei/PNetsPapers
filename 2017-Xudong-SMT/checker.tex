\documentclass[a4paper]{lncs/llncs}

\usepackage[T1]{fontenc}
%\usepackage{geometry}                % See geometry.pdf to learn the layout options. There are lots.
%\geometry{a4paper}                   % ... or a4paper or a5paper or ...
%\geometry{landscape}                % Activate for for rotated page geometry
%\usepackage[parfill]{parskip}    % Activate to begin paragraphs with an empty line rather than an indent
\usepackage{graphicx}

%\usepackage{amsfonts}
%\usepackage{fancyhdr}
%\usepackage{cite}
%\usepackage{ifthen}
%\usepackage{amssymb}
%\usepackage{fancyhdr}
%\usepackage{pifont}
\usepackage{stmaryrd}
\usepackage{mathtools,mathpartir}
\usepackage{proof}
%\usepackage{setspace}
%\usepackage{indentfirst}
\usepackage{amsmath,amssymb,amscd,mathrsfs}
% \usepackage{array,booktabs,arydshln,xcolor}
\usepackage{array,booktabs,xcolor}
\DeclareGraphicsRule{.tif}{png}{.png}{`convert #1 `dirname #1`/`basename #1 .tif`.png}
\usepackage{epsfig,color,subfigure,enumitem}
\newcommand{\TODO}[1]{\textcolor{red}{\textbf{[TODO:#1]}}}
\newcommand{\NOTE}[1]{\textcolor{blue}{\textbf{[NOTE:#1]}}}
\newcommand{\ERIC}[1]{\textcolor{blue}{#1}}
\newcommand{\LUDO}[1]{\textcolor{green}{#1}}
\definecolor{airforceblue}{rgb}{0.26, 0.44, 0.56}
\newcommand{\QIN}[1]{\textcolor{airforceblue}{#1}}
\newcommand{\coloncolon}{{:\hspace{-.2ex}:}}
\makeatletter
\newcommand{\raisemath}[1]{\mathpalette{\raisem@th{#1}}}
\newcommand{\raisem@th}[3]{\raisebox{#1}{$#2#3$}}
\makeatother

\usepackage{macrospNets}

\def\AlgT{\mathcal{T}}
\def\AlgE{\mathcal{E}}
\def\AlgA{\mathcal{A}}
\def\AlgAS{\mathcal{A}_S}
\def\AlgB{\mathcal{B}}
\def\AlgI{\mathcal{I}}
%\newcommand{\set}[1]{\overline{#1}}
\newcommand{\Pred}{\symb{Pred}}
\newcommand{\Post}{\symb{Post}}
%\usepackage[math]{cellspace}
%\setlength\cellspacetoplimit{ 37pt}
%\setlength\cellspacebottomlimit{18pt}

\pagestyle{plain}
% addition to the mathpartir package for red dotted rules,
% that we use for open-transitions
\makeatletter
\def \dotover {\textcolor{red}{\leavevmode\cleaders\hb@xt@ .22em{\hss $\cdot$\hss}\hfill\kern\z@}}
\def \reddottedrule #1#2{\hbox {\advance \hsize by -0.5em
%\sbox0{$\genfrac{}{}{0pt}{0}{#1}{#2}$} \phantom{\copy0} %
 {\ooalign{\vphantom{$\genfrac{}{}{0pt}{0}{#1}{#2}$}\cr\dotover\cr$\genfrac{}{}{0pt}{0}{#1}{#2}$\cr}}}}

 \def \dottedrule #1#2 {
  {\sbox0{$\genfrac{}{}{0pt}{0}{#1}{#2}$}%
    \vphantom{\copy0}%
    \ooalign{%
      \hidewidth
      $\vcenter{\moveright\nulldelimiterspace
        \hbox to\wd0{%
         \xleaders\hbox{\kern.5pt\vrule height 0.4pt width 1.5pt\kern.5pt}\hfill
          \kern-1.5pt
        }%
      }$
      \hidewidth\cr
    \box0\cr}}
}

\let \defaultfraction \mpr@@fraction
\makeatother

%\newtheorem{theorem}{Theorem}[section]
\newtheorem{prop}[theorem]{Proposition}
\newtheorem{cor}[theorem]{Corollary}
%\newtheorem{lemma}[theorem]{Lemma}
\newtheorem{algorithm}[theorem]{Algorithm}
%\newtheorem{remark}[theorem]{Remark}
%\newtheorem{definition}[theorem]{Definition}
%\newtheorem{example}[theorem]{Example}
%\newtheorem{problem}[theorem]{Problem}
%\newtheorem{proof}[theorem]{Proof}

% Macros for the SOS rules and proof trees:
%\newcommand\openrule[2]{\redinfer{#1}{#2}}
\newcommand\openrule[2]{\inferrule*[myfraction=\reddottedrule,center]{#1}{#2}}
%\newcommand\openrule[2]{\inferrule*{#1}{#2}}
%\newcommand\ostate[1]{\triangleleft{\;#1\;}\triangleright}
\newcommand\ostate[1]{\triangleleft{#1}\triangleright}
\newcommand{\sm}[1]{\mbox{\boldmath\small #1}}
%\DeclareMathOperator{\card}{card}
%\DeclareMathOperator{\Flat}{Flat}
\renewcommand{\P}{\mathcal P}
\usepackage{listings}
\lstset{basicstyle=\small}
%\usepackage{algorithm} 
%\usepackage{algpseudocode}  



\begin{document}

This document presumes the definition of Term algebras and
parameterized actions from the pNet
theory definition, that we will not repeat here.

\section{Translations from pNets to SMTLib}
In order to submit satisfiability problems to Z3 (for the predicates
in open transitions), we need to generate SMTlib programs, from the
pNet Algebra presentation and predicates.
More precisely, we need to translate to SMTlib:
\begin{itemize}
  \item the presentation of the action algebra
    (sorts and operators) that is defined for a given language (process calculus, or high level programming language),
    \item for a given pNet, the set of local constants (actions or auxiliary data) that are used in the pLTSs,
    \item for each open transition: the declaration of variables, and
      the predicate (including action expressions).
\end{itemize}

During this translation, in order to guarantee that the generated code
will cause no runtime errors during parsing and execution, we need
to ensure that all objects used in the SMTlib code are properly
declared, and that they are correctly typed.



Note that in principle, such an algebra correspond to a given
high-level language (e.g. a process algebra), and that the algebra
presentation will be defined once and for all in the framework of the
pNet semantics of each specific language.

\section{Algebra presentations}

\begin{definition}
  An Algebra Presentation is a pair $\mathcal{P}=<Sorts,Constrs,Ops>$ where:
  \begin{itemize}
  \item $Sorts$ is a set of Sort names
    \footnote{Later we may want to extend this with Sort constructors,
      like Array or Pair, but this is not needed now}
  \item $Constrs$ is a set of constructor operators, with $Con$ the
    constructor name, and arity: $arity(Con)=n \in \mathbb{N}$,
    and with their signature and their associated selectors,
    of the form: '$Con : (sel_1,sort_1), ... , (sel_n,sort_n) -> sort$'.
    For each argument, the pair $(sel_i,sort_i)$ defines an auxiliary
    operator of name $sel_i$ with signature $sel_i : sort -> sort_i$.
    \item $Ops$ is a set of other (auxiliary) operators, with their
      arity and signature, of the form: $Op : sort_1, ...  sort_n ->
      sort$
      \item $Constrs(sortname), Sels(sortname)$, respectively define the set of
        constructors and selectors of sort $sortname$
  \end{itemize}
  All sorts and operator names must be distinct.\\
  Amongst Constructors, those of arity 0 are called constants, and we
  define $Consts(\mathcal{P}) = \{Con \in Constrs. arity(Con)=0\}$.
\end{definition}


We have a minimal, predefined algebra presentation for all pNets, including
three basic sorts $Bool$, $Action$ and $Int$ and their
operators. Table \ref{Table:predefinedSorts} defines these elements.

In addition, and for convenience, we provide one generic construct for
parameterized actions named FUN, which can accept any number of
arguments of any sort. The result type, though, can only be $Action$,
in order to keep the type-checking simple (see next section).

\begin{table}\caption{\label{Table:predefinedSorts}\leftline{Algebra Presentation: predefined Sorts
  and Operators}}
	\begin{tabular}{p{3cm}p{3cm}p{6cm}}
		\hline\specialrule{0em}{1pt}{1pt}
		Sort & Constructors & Other Operators
                \\\specialrule{0em}{1pt}{1pt}
		\hline\specialrule{0em}{3pt}{3pt}
		Bool    			&
                $\texttt{true},\ \texttt{false}$&
                $\land,\ \lor$
                \\\specialrule{0em}{1pt}{1pt} 
		Action 			&  $\texttt{FUN}$ &
                \\\specialrule{0em}{1pt}{1pt}
		Int 				&
                ${0, \{i, -i\}_{i \in \texttt{Nat}}}$  &
                $- \texttt{(unary)},\ +,\ -
                \texttt{(binary)},\ \times,\ \div, \texttt{etc.}$
                \\\specialrule{0em}{1pt}{1pt}
                \textsl{for any sort} & & $=,\ \ne$
		\\\hline
	\end{tabular}
\end{table}

For a given language, or for a given use-case, the designer can
declare more sorts and operators, using our pNet API.
As an example, a (value-passing) CCS action algebra, where we assume a
single auxiliary value domain ``Data'', can be defined as:

\begin{example}
  \begin{itemize}
  \item $Sorts_{CCS} = \{Action, Channel, Data, Int, Bool\}$
  \item $Constrs_{CCS} =\\
    \texttt{Emit}: 2, \{(Chan\_E:Channel),(Value\_E:Data)\}->Action\\
    \texttt{Receive}: 2, \{(Chan\_R:Channel),(Value\_R:Data)\}->Action\\
    \texttt{Tau}: 0, \{\}-> Action\\
    \texttt{... and all predefined operators}$
  \end{itemize}
\end{example}

\TODO{[Xudong]: Give an example of defining $Tau$ and $Emit$ using the (new
  version of) the API}

\section{Expressions}
Once an algebra presentation is defined, we can construct expressions
of the term algebra, using variables and operators:

\begin{definition}
  Let $\mathcal{V}$ be a set of variables. The Term Algebra
  $\Sigma_{\mathcal{P},\mathcal{V}}$ is the set of:
  \begin{itemize}
  \item variables $x \in \mathcal{V}$
  \item terms (or expressions) $op (arg_1, ... , arg_n)$ with $op$ an operator in
  $\mathcal{P}$ of arity $n$, and each $arg_i$ an expression.
  \end{itemize} 
\end{definition}

\subsection{Constants and variables in pNets and in open transitions}

In addition to the objects defined in the Algebra Presentation, there are
specific objects that are introduced by the pNet construction, and by
the semantic rules used to build the open transitions and their
predicates. This includes, for a pNet $pnet$
\begin{itemize}
\item $Const(pnet)$: Constants from the pLTSs (controllers) transitions: these are
  new constant constructors, usualy of sort Action, local to an
  instance of a controller. The pNet definition requires that all
  these constants are distinct from each other.
\item $SVars(pnet)$: State variables of the controllers. Here also they are required
  to be distinct from those of other controllers.
\item $IVars(pnet)$: Input variables of the controllers.
\item $FVars(pnet,ot)$: Several kinds of ``fresh'' variables, created by application of
  rule Tr2, during the construction of each open transition $ot$:
  Action variables for the behavior of holes and the 
  resulting actions of transitions, variables created during the cloning
  of synchronisation vectors.
\end{itemize}

\def\EPres{\mathcal{P}}
\def\EEnv{\Gamma}

All variable sets above include their sort.
To define the typing rules for expressions and the translation
functions, for any pNet $pnet$ and open transition $ot$, we define an
extended presentation $\EPres_{pnet}$ and environment $\EEnv_{pnet,ot}$ that includes all of the
objects above:

\begin{definition}
  Given an algebra presentation $\mathcal{P}_{pnet}=<Sorts,Constrs,Ops>$ and a pNet $pnet$, we
  construct:
  \begin{itemize}
  \item An extended presentation $\EPres_{pnet}<Sorts,Constrs \cup Const(pnet),Ops>$.
  \item For a given open transition $ot$, an environment
    $\EEnv_{pnet,ot}=SVars(pnet) \cup IVars(pnet) \cup FVars(pnet,ot)$. 
  \end{itemize} 
\end{definition}



\subsection{Well-formed and well-typed expressions}

The purpose of this section is to define static semantic notions that
will guarantee that the translation to the SMT language will be
correct, i.e. will not yield errors at runtime. This includes
well-formness (all sorts, operators, variables are defined, and
expressions respect the arity of operators), and typing rules. 

\begin{definition}
  Given a presentation $\EPres$ (possibly extended) and an environment $\EEnv$:
  \begin{itemize}
  \item $\EEnv$ is well-formed if all sorts in $\EEnv$ are
    defined in $\EPres$
  \item an expression is well-formed if all its operators are defined
    in $\EPres$, and used with the proper arity, and all its
    variables are defined in $\EEnv$
  \item an expression is well-typed if it can be typed by the typing
    rules in table \ref{table:typing-rules}
  \end{itemize}
\end{definition}

The following judgment, and the typing rules in table
\ref{table:typing-rules} can be used to check both the wellformedness
and well-typing of expressions in a pNet or in an open transition,
given the corresponding $\EPres$ and $\EEnv$.

%\begin{table}\caption{\leftline{Typing judgment for Expressions in Open pNets}}
\vspace{1ex}
\noindent
\begin{tabular}{p{5cm}p{7cm}}
		\hline\specialrule{0em}{3pt}{3pt}
%%		$\EEnv \vdash \diamond$ 					& $\EEnv$ is a well-formed environment 					\\\specialrule{0em}{1pt}{1pt}
%%		$\EEnv \vdash A$ 							& A is a well-formed type in $\EEnv$	 					\\\specialrule{0em}{1pt}{1pt}
		$\EPres,\EEnv \vdash M: A$
                & M is a well-formed term of type A in $\EPres,\EEnv$			\\\specialrule{0em}{1pt}{1pt}
		\specialrule{0em}{3pt}{3pt}\hline
	\end{tabular}
%\end{table}	

\begin{table}\caption{\leftline{Type Rules for Open pNets}}
  \label{table:typing-rules}
	\begin{tabular}{p{5cm}p{5cm}p{2.5cm}}
		\hline\specialrule{0em}{3pt}{3pt}
		%% (Env $\varnothing$) 								
		%% & 										
		%% &					\\\specialrule{0em}{1pt}{1pt}
            %% $\dfrac{ }{\varnothing \vdash \diamond}$			
            %% & %$\dfrac{\EEnv \vdash A ~~ x\ }{\varnothing \vdash \diamond}$
            %% &					\\\specialrule{0em}{3pt}{3pt}
		%% (Type Bool) 										
		%% &(Type Action) 						
		%% &(Type Int)			\\\specialrule{0em}{1pt}{1pt}
		%% $\dfrac{\EPres \vdash Bool~~\EEnv \vdash \diamond}{\EPres,\EEnv \vdash Bool}$ 
		%% & $\dfrac{\EPres \vdash Action~~\EEnv \vdash \diamond}{\EPres,\EEnv \vdash Action}$ 
		%% & $\dfrac{\EPres \vdash Int~~\EEnv \vdash \diamond}{\EPres,\EEnv \vdash Int}$        \\\specialrule{0em}{3pt}{3pt}
		(Var x) 										
		& 						
		&			\\\specialrule{0em}{1pt}{1pt}
		$\dfrac{\EEnv \vdash x:A}{\EPres,\EEnv \vdash x:A}$ 
		&  
		&       		\\\specialrule{0em}{5pt}{5pt}
		\multispan{2} (Binary operators, e.g.: $\land, \lor$ for booleans, $+, -, \times, \div, \le, \ge$ for integers, etc.)								
							
		& 			\\\specialrule{0em}{3pt}{3pt}
		$\dfrac{\EPres \vdash BinOp :: ty1, ty1 \rightarrow ty2 ~~\EEnv \vdash x_1:ty1 ~~\EEnv \vdash x_2:ty1}{\EPres,\EEnv \vdash x_1 ~BinOp~ x_2:ty2}$ 
		& 
		&       		\\\specialrule{0em}{3pt}{3pt}
		\multicolumn{2}{l}{(Unary operators, e.g. $\lnot$ for booleans, - for integers)}	
                
		& 			\\\specialrule{0em}{3pt}{3pt}
		$\dfrac{\EPres \vdash UnOp :: ty1 \rightarrow ty2 ~~\EEnv \vdash x:ty1}{\EPres,\EEnv \vdash Unop~x:ty2}$   
		&&       		\\\specialrule{0em}{5pt}{5pt}
		(Polymorphic EQ and NEQ)							
		&					
		&
                                         \\\specialrule{0em}{3pt}{3pt}
		$\dfrac{\EPres \vdash A  ~~\EEnv \vdash x_1:A ~~\EEnv \vdash x_2:A}{\EPres,\EEnv \vdash x_1 = x_2:Bool}$ 
		& 
		$\dfrac{\EPres \vdash A  ~~\EEnv \vdash x_1:A ~~\EEnv \vdash x_2:A}{\EPres,\EEnv \vdash x_1 \ne x_2:Bool}$ 
		
		&       		\\\specialrule{0em}{5pt}{5pt}
		(Overloaded FUN)							
		&					
		& 			\\\specialrule{0em}{3pt}{3pt}
		$\dfrac{\EPres \vdash \texttt{FUN} :: A_1,...,A_n \rightarrow Action ~~\EPres \vdash A_1~~...~~\EPres \vdash A_n ~~\EEnv \vdash x_1:A_1~~...~~\EEnv \vdash x_n:A_n}{\EPres,\EEnv \vdash \texttt{FUN}(x_1,...,x_n):Action}$ 
		& 
		&			\\
		\specialrule{0em}{5pt}{5pt}\hline
	\end{tabular}
\end{table}	

\begin{remark}
  These rules provide a simple type-checking algorithm: if all
  variables in an expression are known in $\EEnv$, then a bottom-up
  application of the rules will decide whether the expression is
  well-typed, and compute the type of each sub-expression.
\end{remark}

\subsection{Map to SMT-LIB language}
The pNet elements, as defined above, can be full translated into
SMT-LIB language, but there are a number of differences in the
structure of the models/langages, so the translation is not trivial.

In this section we shall define separately the translation of the
(extended) algebra presentation for one pNet (so it will be fixed for
the study of one use-case); and the translation of each predicate in
the context of an open-transition of this pNet.

\def\translate{~\lhook\joinrel\longrightarrow}

\vspace{1ex}\noindent
%\begin{table}\caption{\leftline{Mapping}}
	\begin{tabular}{p{3cm}p{9cm}}
		\hline\specialrule{0em}{1pt}{1pt}			
		(Presentation)							
		&								\\\specialrule{0em}{1pt}{1pt}		
		&$Sorts \& Constructors \translate$	declare-datatypes				\\\specialrule{0em}{1pt}{1pt}
		&$Operators \lhook\joinrel\longrightarrow$	declare-function		\\\specialrule{0em}{1pt}{1pt}
		(Predicate)							
		&								\\\specialrule{0em}{1pt}{1pt}
		&$\EEnv ~~\lhook\joinrel\longrightarrow$	declare-const		\\\specialrule{0em}{1pt}{1pt}
		&$Pred ~~\lhook\joinrel\longrightarrow$	 assert		\\\specialrule{0em}{1pt}{1pt}
		&$\texttt{FUN}(x_1,...,x_n)$		\\\specialrule{0em}{1pt}{1pt}
		&$\texttt{pNet expression} ~~\lhook\joinrel\longrightarrow$ SMTLib expression	\\\specialrule{0em}{1pt}{1pt}
		&$...$	\\\specialrule{0em}{1pt}{1pt}
		\specialrule{0em}{1pt}{1pt}\hline
	\end{tabular}
%\end{table}

\smallskip
\ERIC{My plan here is to provide the translation principles as some
  high-level pseudo code, together with precise definition of the
  translation functions of a presentation+environment, and of a
  predicate (including expressions)}

In the following definitions, we use abstract functions corresponding to
SMTlib/Z3 constructs. In practice they can be implemented either as
SMTLib scripting programs, or as calls to the Z3 java API.

\subsection{Presentation Translation}
We define here the translation of the algebra presentation, extended
with the constant operators collected from the pNet. It produces the
declaration of sorts (excepted Bool and Int) with their constructors
and selectors as \texttt{declare-datatypes}, and the declaration of
other operators as \texttt{declare-function} constructs.
For sorts, we must distinguish the case of mutually defined sorts
(e.g. edges and vertices in a graph), that must be declared within a
single \texttt{declare-datatypes} construct. For this we define:

\begin{definition}
Define the (strict) order "is-using" between sorts $S1 is-using S2$ iff $S2$
occurs as the sort of one argument in the constructors of $S1$.
\end{definition}

\TODO{[Xudong: this is certainly obsolete?] At the same time, we have
  several internal lists (actually hash 
  maps) \texttt{exprs}, \texttt{funcDecls}, \texttt{sortDatatypes} to
  store the generated Z3 objects with their names for later proofs.} 

\begin{lstlisting}
Let Pres = <Sorts, Constrs, Ops> an extended presentation
(i.e. including the constants from the pnet),
- define MySorts = Sorts \ {Bool, Int}
- compute the strongly connected components in the graph
  of MySorts with respect to the relation is-using
- for each SCC in this graph, construct: 
  dataype-declaration = TrPresentation (Pres, SCC)
- for each other operator op in Ops, construct:
  function-declaration = TrPresentation (Pres, op)
\end{lstlisting}

%\begin{table}\caption{\leftline{Rules for TrPresentation}}
%  \label{table:TrPresentation}
\noindent
\begin{tabular}{p{5cm}p{5cm}p{2.5cm}}
		\hline\specialrule{0em}{3pt}{3pt}
		\multicolumn{2}{l}{One datatype declaration for each SCC}
                
		& 			\\\specialrule{0em}{1pt}{1pt}
		$\dfrac{name=name(Sort) \qquad constrs=Constrs(name) %\qquad sels=Sels(name)
                  \translate constrs}
                       {SCC \translate
                  \texttt{(declare-datatypes () $name$ (map $\translate constrs$))}}$ 

		&&       		\\\specialrule{0em}{1pt}{1pt}
		\multicolumn{2}{l}{Constants: }	
                
		& 			\\\specialrule{0em}{1pt}{1pt}
		$\dfrac{arity(constr)=0}
                       {constr \translate \texttt{name($constr$)}}$
                
		&&       		\\\specialrule{0em}{1pt}{1pt}
		\multicolumn{2}{l}{Other constructors: }	
                
		& 			\\\specialrule{0em}{1pt}{1pt}
		$\dfrac{n = arity(constr) \neq 0 \qquad |sels|=n
                  \qquad sels =  BuildSels(constr)}
                       {constr \translate \texttt{(name($constr$) . sels)}}$
                       
		&&       		\\\specialrule{0em}{1pt}{1pt}
		\multicolumn{2}{l}{Other operators: }	
                
		& 			\\\specialrule{0em}{1pt}{1pt}
		$\dfrac{op : sort_1, ..., sort_n : sort}
                       {op \translate \texttt{(declare-fun name($op$) ($sortname_1
                           ... sortname_n$) $sortname$)}}$
                       
		&&       		\\\specialrule{0em}{1pt}{1pt}
		\multicolumn{2}{l}{Special case of FUN: }	
                
		& 			\\\specialrule{0em}{1pt}{1pt}
		$\dfrac{}
                       {FUN \translate \texttt{map $\translate CollectFunTypes(pNet)$ }}$
                       
		&&       		\\\specialrule{0em}{1pt}{1pt}
		$\dfrac{}
                       {argstypes \translate \texttt{(declare-fun
                           $BuildFunInstance(argstypes)$ (map name
                           $argstypes$) Action) }}$
                       
		&&       		\\\specialrule{0em}{1pt}{1pt}
		\hline
\end{tabular}
%\end{table}

\smallskip
Where:
\begin{itemize}
  \item the $BuildSels$ function, for a constructor with $n$ arguments,
argument sorts $sort_i$, and selector names $sel_i$, builds the list
$\{(sel_i, sort_i)\}_{i\in[1..n]}$.
\item the $CollectFunTypes$ function collects all possible instances
  of the overloaded FUN argument types found in the pNet, as computed by the typing
  rules; and $BuildFunInstance$ use these argument types as suffixes
  to disembiguate 
\end{itemize}


\begin{example}
  For the CCS presentation above, we would get (in SMTLib syntax):
\begin{lstlisting}
  (declare-datatypes Data ())
  (declare-datatypes Channel ())
  (declare-datatypes ()
      ((Action (Emit (Chan\_E Channel) (Value\_E Data))
               (Receive (Chan\_R Channel) (Value\_R Data))
               Tau )))
  \end{lstlisting}
\end{example}

\TODO{Need to add an exemple with 2 instances of FUN}

 
\subsection{Predicate translation}

\QIN{Each time we submit each open transition to Z3 module, we translate
  its predicate into Z3 language format and send it for satisfiability
  checking. Every term of the predicate is declared as an
  \texttt{assert} in Z3. A constant action or a parameterized
  expression is easy to get from the internal list storing the objects
  while all the variables are not declared at the beginning. So we
  declare them before the submission of a predicate term with the API
  method conducting \texttt{declare-const}.}

The second part of the translation function is called for each open
transition. More precisely we need here:

\begin{lstlisting}
Let Pres = <Sorts, Constrs, Ops> be the extended presentation
and OT = <Leaves, Holes, Pred, Assign> an open transition.
- compute the environment $\EEnv=\EEnv_{pnet,OT}$ collecting
all variables used in OT
- check that all pLTS labels (action, guard, assignement) in
the transition, the OT predicate, and the OT assignments are
well-formed and well-typed
- for each variable $v$ in $\EEnv$, construct:
  define-const = TrPredicate ($\EEnv,v$)
- turn the predicate into conjunctive normal form
- for each conjunct $P_i$ in the predicate, construct:
  assert = TrPredicate $(P_i)$
\end{lstlisting}

\noindent
\begin{tabular}{p{5cm}p{5cm}p{2.5cm}}
		\hline\specialrule{0em}{3pt}{3pt}
		\multicolumn{2}{l}{Variables}
                
		& 			\\\specialrule{0em}{1pt}{1pt}
		$\dfrac{\EEnv \vdash x:A }
                       {x \translate \texttt{(declare-const x A)}}$
                  
		&&       		\\\specialrule{0em}{1pt}{1pt}
		\multicolumn{2}{l}{Predicate conjunct: }	
                
		& 			\\\specialrule{0em}{1pt}{1pt}
		$\dfrac{}
                       {P_i \translate \texttt{(assert ... }}$
                
		&&       		\\\specialrule{0em}{1pt}{1pt}
		\multicolumn{3}{p{12cm}}{To be completed with translation of
                expressions, including special cases for EQ, NOT\_EQ,
                and FUN }	
                
		 			\\\specialrule{0em}{1pt}{1pt}
		\hline
\end{tabular}
%\end{table}


\section{Submission to Z3}
\TODO{Do we still need that ? Maybe rather something linking my
  "abstract SMTLib functions" with both SMTLib python code, and Z3 API
  methods.}


\subsection{Z3 Notations}
Our semantic has already got enough informations for Z3 checking. Z3 has a number of notations for various usages. We only list the notations we needed in our algorithm as follows:
\TODO{Finish the description.}
\begin{enumerate}
\item \texttt{declare-datatypes}
\item \texttt{declare-function}
\item \texttt{declare-const}
\item \texttt{assert}
\end{enumerate}



\subsection{Other works}
\paragraph{Quantifier}


\end{document}
