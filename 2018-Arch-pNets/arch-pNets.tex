% This is LLNCS.DOC the documentation file of
% the LaTeX2e class from Springer-Verlag
% for Lecture Notes in Computer Science, version 2.4
\documentclass{llncs}
\usepackage{llncsdoc}

%\usepackage{a4wide}

\usepackage{graphicx}
\usepackage{verbatim}


\let\proof\relax   
\let\endproof\relax

\usepackage{amsmath,amsthm,amscd}
\usepackage{amssymb}    % for blackboard B
\usepackage{rotating}   % to rotate the figures
\usepackage{mathpartir} % math paragraph + inference rules
\usepackage{mathtools}  % for align, multline etc.
\usepackage{caption}
\usepackage{xifthen}
\usepackage{stmaryrd} % double brackets (\llbracket, \rrbracket)

%% \usepackage[justification=centering,belowskip=-10pt,aboveskip=0pt]{caption}
%% \setlength{\intextsep}{10pt plus 2pt minus 2pt}
%% \setlength\abovedisplayskip{0pt}

\usepackage{wrapfig}
\usepackage{subcaption}
\usepackage{hyperref}

\usepackage{soul}
\usepackage[colorinlistoftodos,bordercolor=white]{todonotes}
%% \usepackage[disable,colorinlistoftodos,bordercolor=white]{todonotes}
\usepackage{macrospNets}

\newcommand{\Simon}{\\\hfill\mdash Simon}

\newcommand{\noteSB}[2][color=green!40, size=\tiny]{\todo[#1]{{#2}\Simon}}
\newcommand{\noteSBin}[2][inline,color=green!40]{\todo[#1]{{#2}\Simon}}
\newcommand{\todoSB}[2][color=green!40, size=\tiny]{\todo[#1]{\textbf{To-do Simon:} {#2}}}
\newcommand{\todoSBin}[2][inline,color=green!40]{\todo[#1]{\textbf{To-do Simon: } {#2}}}
\newcommand{\Ludo}{\\\hfill\mdash Ludo}
\newcommand{\noteLH}[2][color=orange!40, size=\tiny]{\todo[#1]{{#2}\Ludo}}
\newcommand{\noteLHin}[2][inline,color=orange!40]{\todo[#1]{{#2}\Ludo}}
\newcommand{\todoLH}[2][color=orange!40, size=\tiny]{\todo[#1]{\textbf{To-do Ludo:} {#2}}}
\newcommand{\todoLHin}[2][inline,color=orange!40]{\todo[#1]{\textbf{To-do Ludo: } {#2}}}

\newcommand{\noteIn}[2][inline,color=black!20]{\todo[#1]{{#2}}}

\newcommand{\add}[2][Added]{\todo[color=blue!20, size=\tiny]{#1}{\color{blue}#2}}
%% \newcommand{\remove}[2][Removed]{\todo[color=red!20, size=\tiny]{#1}}
\newcommand{\remove}[2][Removed]{\todo[color=red!20, size=\tiny]{#1}{\color{red}\st{#2}}}
\newcommand{\removeIn}[2][Removed]{\todo[color=red!10, inline]{{\color{red}#2}#1}}
%% \newcommand{\replace}[3][Replaced]{\todo[color=blue!20, size=\tiny]{#1}{\color{blue}#3}}
\newcommand{\replace}[3][Replaced]{\todo[color=blue!20, size=\tiny]{#1}{\color{blue}#3}{\color{red}\st{#2}}}

\newcommand{\addSB}[1]{\add[Added by Simon]{#1}}
\newcommand{\removeSB}[1]{\remove[Removed by Simon]{#1}}
\newcommand{\removeSBin}[1]{\removeIn[Removed by Simon]{#1}}
\newcommand{\replaceSB}[2]{\replace[Replaced by Simon]{#1}{#2}}

\newcommand{\defn}[1]{Def.~\ref{defn:#1}}
\newcommand{\defntwo}[2]{Defs~\ref{defn:#1} and \ref{defn:#2}}
\newcommand{\fig}[1]{Fig.~\ref{fig:#1}}
\newcommand{\figs}[2]{Fig.~\ref{fig:#1} and~\ref{fig:#2}}
\newcommand{\tab}[1]{Tab.~\ref{tab:#1}}
\newcommand{\eq}[1]{(\ref{eq:#1})}
\newcommand{\res}[1]{(\ref{res:#1})}
\newcommand{\ex}[1]{Ex.~\ref{ex:#1}}
\newcommand{\exs}[2]{Ex.~\ref{ex:#1} and~\ref{ex:#2}}
\newcommand{\secn}[1]{Sect.~\ref{secn:#1}}
\newcommand{\rem}[1]{Rem.~\ref{rem:#1}}
\newcommand{\lem}[1]{Lem.~\ref{lem:#1}}
\newcommand{\cor}[1]{Cor.~\ref{cor:#1}}
\newcommand{\thm}[1]{Th.~\ref{thm:#1}}
\newcommand{\axs}[1]{Ax.~\ref{ax:#1}}
% \newcommand{\ax}[2]{\ref{ax:#1}.\ref{ax:#1:#2}}
\newcommand{\ax}[1]{Ax.~\ref{ax:#1}}
\newcommand{\prop}[1]{Prop.~\ref{prop:#1}}
\newcommand{\alg}[1]{Alg.~\ref{alg:#1}}
\newcommand{\asmp}[1]{Ass.~\ref{asmp:#1}}

% %%%%%%%%%%%%%%%%%%%%%

\newcommand{\cA}{\ensuremath{\mathcal{A}}}
\newcommand{\fA}{\ensuremath{\mathsf{A}}}
\newcommand{\sA}{\ensuremath{\mathbb{A}}}
\newcommand{\cB}{\ensuremath{\mathcal{B}}}
\newcommand{\fB}{\ensuremath{\mathsf{B}}}
\newcommand{\sB}{\ensuremath{\mathbb{B}}}
\newcommand{\cC}{\ensuremath{\mathcal{C}}}
\newcommand{\sC}{\ensuremath{\mathbb{C}}}
\newcommand{\cD}{\ensuremath{\mathcal{D}}}
\newcommand{\fD}{\ensuremath{\mathsf{D}}}
\newcommand{\sD}{\ensuremath{\mathbb{D}}}
\newcommand{\cE}{\ensuremath{\mathcal{E}}}
\newcommand{\sE}{\ensuremath{\mathbb{E}}}
\newcommand{\cF}{\ensuremath{\mathcal{F}}}
\newcommand{\cG}{\ensuremath{\mathcal{G}}}
\newcommand{\cH}{\ensuremath{\mathcal{H}}}
\newcommand{\cI}{\ensuremath{\mathcal{I}}}
\newcommand{\sI}{\ensuremath{\mathbb{I}}}
\newcommand{\cM}{\ensuremath{\mathcal{M}}}
\newcommand{\cN}{\ensuremath{\mathcal{N}}}
\newcommand{\sN}{\ensuremath{\mathbb{N}}}
\newcommand{\cP}{\ensuremath{\mathcal{P}}}
\newcommand{\sP}{\ensuremath{\mathbb{P}}}
\newcommand{\cQ}{\ensuremath{\mathcal{Q}}}
\newcommand{\sQ}{\ensuremath{\mathbb{Q}}}
\newcommand{\cR}{\ensuremath{\mathcal{R}}}
\newcommand{\sR}{\ensuremath{\mathbb{R}}}
\newcommand{\cS}{\ensuremath{\mathcal{S}}}
\newcommand{\sS}{\ensuremath{\mathfrak{S}}}
\newcommand{\cT}{\ensuremath{\mathcal{T}}}
\newcommand{\fT}{\ensuremath{\mathsf{T}}}
\newcommand{\sT}{\ensuremath{\mathbb{T}}}
\newcommand{\cV}{\ensuremath{\mathcal{V}}}
\newcommand{\fV}{\ensuremath{\mathsf{V}}}
\newcommand{\sV}{\ensuremath{\mathbb{V}}}
\newcommand{\sZ}{\ensuremath{\mathbb{Z}}}

\newcommand{\mdash}[1][]{---#1}
\newcommand{\ndash}{--}
\newcommand{\ie}[1][\ ]{i.e.#1}
\newcommand{\etc}[1][\ ]{etc.#1}
\newcommand{\eg}[1][\ ]{e.g.#1}
\newcommand{\cf}[1][\ ]{cf.#1}
\newcommand{\wrt}[1][\ ]{w.r.t.#1}
\newcommand{\resp}[1][\ ]{resp.#1}

\newcommand{\bydef}[1]{\ensuremath{\stackrel{\mathit{\scriptscriptstyle def}}{#1}}}
\newcommand{\suchthat}{\ensuremath{\,|\,}}
\newcommand{\rightsuchthat}{\ensuremath{\,\right|\,}}
\newcommand{\leftsuchthat}{\ensuremath{\,\left|\,}}
\newcommand{\setdef}[2]{\ensuremath{\{{#1}\,|\,{#2}\}}}
\newcommand{\setdefb}[2]{\ensuremath{\bigl\{{#1}\,\bigl|\,{#2}\bigr.\bigr\}}}
\newcommand{\Setdef}[2]{\ensuremath{\Big\{{#1}\,\Big|\,{#2}\Big\}}}
\newcommand{\goesto}[2][]{\ensuremath{\xrightarrow[#1]{#2}}}
\newcommand{\notgoesto}[2][]{\ensuremath{\not\xrightarrow[#1]{\ \,#2}}}
\newcommand{\non}[1]{\ensuremath{\overline{#1}}}

\newcommand{\true} {\ensuremath{\mathtt{t\!t}}}
\newcommand{\false}{\ensuremath{\mathtt{f\!f}}}
\newcommand{\noop} {\ensuremath{\emptyset}} % \mathsf{skip}

\newcommand{\ordbool}{\ensuremath{\sB^{<}}}
\newcommand{\data}{\ensuremath{\sD}}
\newcommand{\signature}{\ensuremath{\Sigma}}
\newcommand{\variables}{\ensuremath{\cV}}
\newcommand{\Talg}{\ensuremath{\cT_{\signature,\variables}}}
\newcommand{\actions}[1]{\ensuremath{\cA_{#1}}}
\newcommand{\guards}[1]{\ensuremath{\ordbool_{#1}}}
\newcommand{\exprs}[1]{\ensuremath{\sE_{#1}}}
\newcommand{\assigns}[1]{\ensuremath{\sA_{#1}}}
\newcommand{\valuations}[1]{\ensuremath{\data^{#1}}}
\newcommand{\val}[3][]{\ensuremath{#1{\sigma}^{#2}_{#3}}}

\newcommand{\primeit}[1]{#1'}
\newcommand{\doubleprimeit}[1]{#1''}


\newcommand{\export}[1][]{\ensuremath{\varepsilon_{#1}}}
\newcommand{\valdiff}[2]{\ensuremath{#1 \triangle #2}}
\newcommand{\supp}[1]{\ensuremath{\mathrm{supp}(#1)}}
\newcommand{\semopen}[1]{\ensuremath{[{#1}]}}
\newcommand{\semclosed}[1]{\ensuremath{\llbracket{#1}\rrbracket}}
\newcommand{\reachable}[1]{\ensuremath{\mathit{reachable}({#1})}}
\newcommand{\IMextend}[2]{\ensuremath{#1 \ltimes #2}}
\newcommand{\arcomp}{\oplus}
\newcommand{\arequiv}{\equiv}
\newcommand{\Arcomp}{\Bigoplus}
\newcommand{\expmix}{\wedge}
\newcommand{\order}{\leqslant}

\makeatletter
\newcommand{\doubletilde}[1]{{%
  \mathpalette\double@tilde{#1}%
}}
\newcommand{\double@tilde}[2]{%
  \sbox\z@{$\m@th#1\tilde{#2}$}%
  \ht\z@=.9\ht\z@
  \tilde{\box\z@}%
}
\makeatother

\newcounter{tempctr}
\newcommand{\breakenumistart}{%
  \setcounter{tempctr}{\value{enumi}}%
  \end{enumerate}%
}
\newcommand{\breakenumiend}{%
  \begin{enumerate}%
  \setcounter{enumi}{\value{tempctr}}%
}

% %%%%%%%%%%%%%%%%%%%%%

\usepackage{tikz}
\usetikzlibrary{calc}
\usetikzlibrary{arrows,shapes,automata,petri}
  \tikzset{
  place/.style={
    circle,
    thick,
    draw=blue!75,
    fill=blue!20,
    minimum size=6mm
  },
  transition/.style={
    rectangle,
    thick,
    fill=black,
    minimum width=8mm,
    inner ysep=2pt
    },
    arc/.style = {
      decoration=
      {markings,mark=at position #1 with {circle;}
      },
      postaction={decorate,draw}},
  }   

\let\llncssubparagraph\subparagraph
\let\subparagraph\llncssubparagraph

\begin{document}
\graphicspath{{figures/}}

\title{Architectures and Open pNets}

\author{%
Simon~Bliudze\inst{1}
\and
Ludovic~Henrio\inst{2}
\and
Eric~Madelaine\inst{3}
\and
...
}

\institute{%
  INRIA Lille -- Nord Europe, Villeneuve d'Ascq, France\\
  \email{simon.bliudze@inria.fr}
\and
\and
}


\maketitle

\begin{abstract}
  We extend the theory of architectures with a general
  mechanism for handling various types of data transfer.  We
  encode this extended model into the open pNet semantic
  model and show how this encoding can be used to verify
  that an architecture does, indeed, enforce its
  characteristic property.

\keywords{}
\end{abstract}

%****************************************************************
%****************************************************************

\section{Introduction}
\label{secn:introduction}

%****************************************************************
%****************************************************************

\section{Preliminaries}
\label{secn:preliminaries}

%****************************************************************

\subsection{Notations}
\label{secn:notations}

We extensively use indexed structures
over some countable indexed sets, which are equivalent to mappings over
the countable set. 
Thus, 
$a_i^{i\in I}$
%, or equivalently  $(i\mapsto a_i)^{i\in I}$
denotes a family of elements $a_i$ indexed over the
set $I$. % Such a family
% is equivalent to the mapping $(i\mapsto a_i)^{i\in I}$.
% To specify the set over which the structure is indexed,
% indexed structures are always denoted with an exponent of the form $i\in I$
% (arithmetic only appears in the indexes if necessary).
This notation defines both $I$ the set over which the family is
indexed (called \emph{range}), and $a_i$ the elements of the family.
E.g., $a^{i\in\{3\}}$ is the mapping with a single entry $a$ at index
$3$ ; abbreviated $(3\mapsto a)$ in the following.
When this is not
ambiguous, we shall use notations for sets, and typically write
``indexed set over I'', even though formally we should speak of multisets; and
write $x\in a_i^{i\in I}$ to mean $\exists i\in I.\, x=a_i$.  An empty
family is denoted $\emptyset$. We use a classical overline notation, \eg $\overline{a}$, 
for a family, where the indexing set is irrelevant. Finally, we use $\uplus$ to denote 
the disjoint union on
indexed sets.

\noteIn{states looks ok, but initial state is $s_0$ in pNet and not in archi : $s^0$.  Exponent 
position was removed from pNet because we found it complex with indexed set notation but 
could be ok.
\Ludo

I am using exponent notation for ``modifiers'', reserving\mdash
as much as possible\mdash index notation for enumeration.  I
think this can be solved, where necessary, by using parentheses,
\eg $(s^0_i)^{i \in I}$
\Simon
}

%****************************************************************

\subsection{Term algebra}
\label{secn:terms}

\todoLHin{Remove term algebra or simplify?}
\todoLHin{universal domain $\implies$ remove sort}
\todoLHin{replace assignments by $e_i$

  In fact, in the pNet action terms, you need expressions, not
  assignements.  Therefore assuming that all expressions are
  assignments will not work, so the right approach is to have
  expressions (for pNets) and assignments (for BIP) separately.
  \Simon
%
}

\noteSBin{In Archi we have $a$, whereas in pNets we have $\alpha$, where, approximately $\alpha = 
a(x_1..x_n)$, except that $a$ is port name in archi and action name in pNets.
Is this true? can we unify better? or is it sufficient?
\Ludo

$a$ is a set of ports, but otherwise, yes, this is true.  I think
this is coherent, since, in archi, we have the ``interaction
part'', \ie only ports, plus the data part, \ie variables,
guards, expressions.  In pNets, these are combined.  So, it seems
ok that notations differ\mdash ``uniformisation'' will be
achieved by encoding archi $\rightarrow$ pNets.
}

\noteLHin{I believe we should simplify this paragraph about param actions}
The pNet model in \secn{pNets} relies on a notion of parameterised actions, that are
symbolic expressions using data types and variables. As we aim at encoding the low-level 
behaviour of possibly very different
programming languages, we do not want to impose one specific algebra
for denoting actions, nor any specific communication mechanism. So we
leave unspecified the constructors of the algebra that will allow building
expressions and actions. Moreover, we use a generic {\em action interaction}
mechanism, based on (some sort of) unification between two or more action
expressions, to express various kinds of communication or
synchronisation mechanisms.

%\def\Talg{\mathcal{T}_{\Sigma,\P}}  % Moved to the header

\noteSBin{proposed notations:
  
  $g$ are boolean expressions;
  $e_i$ are updates/assignments
  \Ludo
 
  In fact, I am not sure whether I have correctly understood this
  suggestion: do you mean, \eg using $g$ to denote the set of
  Boolean expressions or individual Boolean expressions? Using
  lower case letters to denote sets might be problematic, since
  we (at least I) use them a lot to denote instances, as in
  \[q \goesto{a, g, e} q'\,.\]
  In fact, I even use capital letters: in the definition of
  interaction model composition, I use
  \[\cdot \goesto{a, G, E} \cdot\]
  in the conclusion of the rule.  However, for this latter I can
  easily change the notation.

  I have set up macros for, general and Boolean expressions,
  actions, variables \etc[,] which I have substituted everywhere,
  where appropriate in the following paragraph.
}

  \begin{tabular}{
      @{}p{0.15\columnwidth}@{}
      p{0.35\columnwidth}@{}
      p{0.5\columnwidth}@{}
    }
    {\bf Output} & {\bf Command} & {\bf Definition}
    \\
    $\data$ &
    {\ttfamily\textbackslash data} &
    {\ttfamily\textbackslash sD}
    \\
    \signature &
    {\ttfamily\textbackslash signature} &
    {\ttfamily\textbackslash Sigma}
    \\
    \variables &
    {\ttfamily\textbackslash variables} &
    {\ttfamily\textbackslash cV}
    \\
    \Talg &
    {\ttfamily\textbackslash Talg} &
    {\ttfamily\textbackslash cT\_\{\textbackslash signature,\textbackslash variables\}}
    \\
    \actions{\#1} &
    {\ttfamily\textbackslash actions\{\#1\}} &
    {\ttfamily\textbackslash cA\_\{\#1\}}
    \\
    \guards{\#1} &
    {\ttfamily\textbackslash guards\{\#1\}} &
    {\ttfamily\textbackslash sB\^{}\{<\}\{\#1\}}
    \\
    \exprs{\#1} &       
    {\ttfamily\textbackslash exprs\{\#1\}} &
    {\ttfamily\textbackslash sE\_\{\#1\}}
    \\
    \assigns{\#1} &
    {\ttfamily\textbackslash assigns\{\#1\}} &
    {\ttfamily\textbackslash sA\_\{\#1\}}
    \\
    \valuations{\#1} &
    {\ttfamily\textbackslash valuations\{\#1\}} &
    {\ttfamily\textbackslash data\^{}\{\#1\}}
    \\[6pt]
    \val{\#2}{\#3} &
    {\ttfamily\textbackslash val\{\#2\}\{\#3\}} &
    {\ttfamily\#1\{\textbackslash sigma\}\^{}\{\#2\}\_\{\#3\}}
    \\[6pt]
    \val[\tilde]{\#2}{\#3} &
    {\ttfamily\textbackslash val[\textbackslash tilde]\{\#2\}\{\#3\}} &
    {\ttfamily\#1\{\textbackslash sigma\}\^{}\{\#2\}\_\{\#3\}}
  \end{tabular}
\noteSBin{commands above (tabular generates an error inside a
  note)}
  
Formally, we assume the existence of a term algebra $\Talg$,
where $\signature$ is the signature of the data and action constructors,
and $\variables$ a set of \emph{variables}. Within $\Talg$, we distinguish a set of
\emph{data expressions} $\exprs{\variables}$ \replaceSB{, including}{and} a set of \emph{Boolean
expressions} \todoLH{$\mathcal{B} \to g$}
$\guards{\variables}$ ($\guards{\variables}\subseteq\exprs{\variables}$).
On top of $\exprs{\variables}$ we build the \emph{action algebra}
$\actions{\variables}$, with $\actions{\variables}\subseteq\Talg$ \replaceSB{,}{and}
$\exprs{\variables}\cap\actions{\variables} = \emptyset$;
naturally action terms will use data expressions as subterms.
\addSB{In addition to the above notations, we will use
\[
\assigns{\variables} \bydef{=}
\Setdef{(x_i := e_i)^{i \in I}}{
  x_i^{i \in I}\subseteq \variables,
  e_i^{i \in I}\subseteq \exprs{\variables},
  I \text{ is a finite index set}
}
\]
to denote the set of \emph{variable assignments}.}

To be able to reason about the data flow between pLTSs, we
distinguish \emph{input variables} of the form $?x$ within terms; the function
$\vars(t)$ identifies the set of variables in a term
$t\in\Talg$, and $iv(t)$ returns its input variables.
Action algebras can \replaceSB{encode naturally }{naturally encode }usual point-to-point message passing calculi (using 
$a(?x_1,\dots,?x_n)$ for inputs, $a(v_1,\dots,v_n)$ for outputs), but it also allows
for more general mechanisms, like gate negotiation in Lotos, or broadcast
communications.

\addSB{%
%
  Finally, we assume the existence of a universal data domain
  given as a partially-ordered set $(\data, \order)$, potentially
  encompassing several copies of any given data type with
  different orders.  For example, we assume that $(\data,
  \order)$ comprises both the unordered set of Booleans $\sB =
  (\{\true, \false\}, \emptyset)$ and the naturally ordered one
  $\sB^< = (\{\true, \false\}, \{\false < \true\})$, and
  similarly for integer and real numbers; as well as the set of
  intervals ordered by inclusion.

  For a set of variables $V \subseteq \variables$, we denote
  $\valuations{V} \bydef{=} \{\val{}{}: V \rightarrow \data\}$
  the set of \emph{valuations} of the variables in $V$.
  Valuations extend canonically to expressions.  Thus
  $\val{}{}(x)$ and $\val{}{}(e)$ denote, respectively, the
  values of the variable $x$ and the expression $e$ under the
  valuation $\val{}{}$.  For a valuation $\val{}{} \in
  \valuations{V}$ and an assignment $(x_i := e_i)^{i \in I} \in
  \assigns{V}$, we denote $\val{}{}[(x_i := e_i)^{i \in I}]$ the
  valuation defined by
  \[
  \val{}{}\bigl[(x_i := e_i)^{i \in I}\bigr](x) \bydef{=}
  \begin{cases}
    \val{}{}(x), & \text{if } x \not\in x_i^{i \in I}\,,\\
    \val{}{}(e_i), & \text{if } x = x_i, \text{for some } i \in I\,.
  \end{cases}
  \]
  For two valuations $\val{1}{}, \val{2}{} : V \rightarrow
  \data$, we denote \[\valdiff{\val{1}{}}{\val{2}{}} \bydef{=}
  \setdefb{v \in V}{\val{1}{}(v) \neq \val{2}{}(v)}\] the set of
  variables that are assigned different values by the two
  valuations.  As usual, we write $\val{1}{} \order \val{2}{}$
  iff $\val{1}{}(x) \order \val{2}{}(x)$, for all $x \in V$.  An
  expression $e \in \exprs{V}$ is called \emph{monotonic} if, for
  any two valuations $\val{1}{}, \val{2}{} \in \valuations{V}$,
  $\val{1}{} \order \val{2}{}$ implies $\val{1}{}(e) \order
  \val{2}{}(e)$.
%
}

%****************************************************************
%****************************************************************

\section{The theory of architectures with data}
\label{secn:archs}

%****************************************************************

\subsection{Components and composition}
\label{sec:components}

\noteSBin{%
To guarantee preservation of properties guards and expressions
have to be \emph{monotonic}.  In the case of a trivial order, \ie
when nothing is comparable, \emph{any guard or expression is
  monotonic}.  Thus, this is not really a constraint!

Do we put this in the definition of expressions and guards or in
the definitions of components and interaction models?
}

\begin{definition}[Components]
  \label{defn:component}
  A \emph{component} is a tuple $(Q, q^0, V, \val{0}{}, P,
  \export{}, \goesto{})$, where
  \begin{itemize}
  \item $Q$ is a set of \emph{states}, with $q^0 \in Q$ the
    \emph{initial state}, 
  \item $V$ is a set of \emph{component variables},
  \item $\val{0}{} : V \rightarrow \data$ is an \emph{initial
    valuation} of the component variables, 
  \item $P$ is a set of \emph{ports}, with
    \noteSB{Added this}\hl{$\export : P \rightarrow 2^V$ 
    the \emph{export function},}
  \item $\goesto{}\, \subseteq
    Q \times 2^P \times \guards{V} \times \exprs{V} \times Q$
%
    is a \emph{transition relation}, with transitions
    labelled by triples consisting of an \emph{interaction}
    $\emptyset \neq a \subseteq P$, a monotonic Boolean \emph{guard} $g \in
    \guards{V}$ and a monotonic \emph{update expression} \todoSB{$e \rightarrow u$}\hl{$e \in
    \exprs{V}$}.
  \end{itemize}
%
  We call \noteSB{Maybe rather $(P, \export)$}\hl{the pair of sets $(V,P)$} the \emph{interface} of the
  component.\,\footnote{%
%
  In practice, component variables are split in two sub-sets:
  local and exported variables.  Only exported variables belong
  to the component interface and can be used for the interactions
  with other components (see \defn{im}).  However, in the context
  of this paper we can simplify by omitting this separation.
%
  }
%
  Notations $q \goesto{a} q'$ and $q \goesto{a}$ are as usual;
  for a component $B$, we denote $Q_B$, $q^0_B$, $P_B$, $V_B$ and
  $\val{0}{B}$ the corresponding constituents of $B$.  We will
  skip the index on the transition relations $\goesto{}$, since it
  is always clear from the context.
\end{definition}

\begin{definition}[Component semantics]
  \label{defn:comp:semantics}
  The \emph{open semantics} of a component $B = (Q, q^0, V,
  \val{0}{}, P, \export, \goesto{})$ is given by the LTS $\semopen{B} = (S,
  s^0, \goesto{})$, where $S = Q \times \valuations{V}$, $s^0 =
  (q^0, \val{0}{})$ and $\goesto{}$ is the minimal transition
  relation satisfying the rule
  %
  \begin{equation}
    \label{eq:comp:semantics}
    \infer{
      q \goesto{a, g, e} q'
      \and
      \val{}{} \models g
      \and
      \val[\primeit]{}{} = \tilde{\val{}{}}[e]
      \and
      \valdiff{\val{}{}}{\tilde{\val{}{}}} \subseteq \export(a)
    }{
      (q, \val{}{}) \goesto{a,\tilde{\val{}{}}} (q', \val[\primeit]{}{})
    }
    \,.
  \end{equation}
  %
  \todoSB{Do we actually need this?}
  \hl{The \emph{closed semantics} of $B$ is given by the LTS
  $\semclosed{B} \bydef{=}\reachable{\semopen{B}}$, comprising
  only the reachable states of $\semopen{B}$.}
\end{definition}

The use of the intermediate valuation $\tilde{\val{}{}}$ in the
conclusion and the third premise of rule \eq{comp:semantics}
allows some of the variables to get new values before the
transition is actually fired.  Thus the component is \emph{open}
to the exchange of data with its environment.  However, the
fourth premise in \eq{comp:semantics} restricts the set of
variables, which can be affected by such data transfer, to those
that are exported through the ports participating in the
interaction.

\begin{definition}[Interaction model \& composition]
  \label{defn:im}
  Consider a finite set of components $\cB = (Q_i, q^0_i, V_i,
  \val{0}{i}, P_i, \export[i], \goesto{})^{i \in I}$, such that
  all their respective components (\ie $Q_i$, $P_i$, and $V_i$)
  are pairwise disjoint, \ie $\forall i \neq j,\ Q_i \cap Q_j =
  P_i \cap P_j = V_i \cap V_j = \emptyset$.  Let $P = \bigcup_{i
  \in I} P_i$ and $V = \bigcup_{i \in I} V_i$.
%
  For a set of ports $a \subseteq P$, we denote $\supp{a}
  \bydef{=} \setdef{i \in I}{a \cap P_i \neq \emptyset}$ the
  \emph{support} of $a$,
  \noteSB{Updated this}\hl{and $V_a \bydef{=} \bigcup_{i \in
    \supp{a}} \export[i](a \cap P_i)$ the set of component
  variables exported for $a$.}

  An \emph{interaction model over \hl{$(V,P)$}} is a set $\Gamma
  \subseteq 2^P \times \guards{V} \times \exprs{V}$, such that,
  for any $(a, g, e) \in \Gamma$, the guard and update expression
  associated to the \emph{interaction} $a$ satisfy, respectively,
  $g \in \guards{V_a}$ and $e \in \exprs{V_a}$.\footnote{%
%
    Notice that this definition allows $(\emptyset, \true,
    \noop)$ and $(\emptyset, \false, \noop)$ to be included in
    $\Gamma$.
%
  }
  %% We call the set of ports $P$ the \emph{domain} of the
  %% interaction model.

  The \emph{composition of $\cB$ with the interaction model
    $\Gamma$} is the component
  $\Gamma(\cB) = (Q, q^0, V, \val{0}{}, P, \export, \goesto{})$,
  where
%
  $Q = \prod_{i \in I} Q_i$;
%
  $q^0 = (q_i^0)^{i \in I}$;
%
  $\val{0}{}: V \rightarrow \data$ is such that, for any $v \in
  V_i$, $\val{0}{}(v) = \val{0}{i}(v)$;
%
  $\export : P \rightarrow 2^V$ is such that, for any $p \in
  P_i$, $\export(p) = \export[i](p)$;
%
  and $\goesto{}$ is the minimal transition relation satisfying
  the rule
%
  \begin{gather}
%%     \label{eq:im:empty}
%%     \infer{
%%       q_j \goesto{\emptyset, g, e} q_j'
%%       \and
%%       \forall i \neq j, q_i = q_i'
%%     }{
%%       (q_i)^{i \in I} \goesto{\emptyset, g, e} (q_i')^{i \in I}
%%     }\,,
%% %
%%     \\[0.5\baselineskip]
    \label{eq:im:int}
%
    \infer{    
      (a, g, e) \in \Gamma
      \and
      a \neq \emptyset
      \and
      \forall i \in \supp{a}, q_i \goesto{a \cap P_i, g_i, e_i} q_i'
      \and
      \forall i \not\in \supp{a}, q_i = q_i'
%
      \\\\
      \textstyle
%
      G = g \land \bigwedge_{i \in \supp{a}} g_i
      \and
      E = e; e_i^{i \in \supp{a}}
    }{
      (q_i)^{i \in I} \goesto{a, G, E} (q_i')^{i \in I}
    }\,.
  \end{gather}
\end{definition}

Below, when speaking of a set of components $\cB$, we will always
assume that it satisfies all the assumptions of \defn{im}.
%
For an interaction $a$, we will abuse the notation by writing
$\supp{a}$ to also denote the set $\setdef{B \in \cB}{a \cap P_B
  \neq \emptyset}$.  The precise meaning of this notation will
always be clear from the context.

%****************************************************************
\subsection{Graphical representation of composed systems}
\label{secn:connectors}

%****************************************************************
\subsection{Architectures}
\label{secn:archi}

\begin{definition}[Architecture]
  \label{defn:arch}
  An \emph{architecture} is a tuple $A = (\cC, V_A, P_A, \Gamma)$,
  where $\cC$ is a finite set of \emph{coordinating components}
  with pairwise disjoint sets of ports and variables, such that
  $\bigcup_{C \in \cC} P_C \subseteq P_A$ and
  $\bigcup_{C \in \cC} V_C \subseteq V_A$, and
  $\Gamma \subseteq 2^{P_A} \times \guards{V_A} \times \exprs{V_A}$
  is an interaction model over \hl{$(V_A, P_A)$}.
\end{definition}

\begin{definition}[Application of an architecture]
  \label{defn:arch:application}
  Let $A = (\cC, V_A, P_A, \Gamma)$ be an architecture and let $\cB$
  be a set of components, such that
%
  \begin{align}
    \bigcup_{B \in \cB} V_B \cap \bigcup_{C \in \cC} V_C = \emptyset\,,
    &&
    V_A \subseteq V \bydef{=} \bigcup_{B \in \cB \cup \cC} V_B\,,
    \\
    \bigcup_{B \in \cB} P_B \cap \bigcup_{C \in \cC} P_C = \emptyset\,,
    &&
    P_A \subseteq P \bydef{=} \bigcup_{B \in \cB \cup \cC} P_B\,,
  \end{align}
%
  and
  %% , denoting, for any $(a, g, e) \in \Gamma$, $V_a \bydef{=}
  %% \bigcup_{B \in \supp{a}} \export[B](a \cap P_B)$, we have
  $g \in \guards{V_a}$ and
  $e \in \exprs{V_a}$ (see \defn{im} for the notation $V_a$).
%
  The \emph{application of the architecture $A$ to the set of
  components $\cB$} is the component $ A(\cB) \bydef{=}
  (\IMextend{\Gamma}{P})(\cC \cup \cB)$, where
%
  \begin{equation}
    \label{eq:im:extension}
    \IMextend{\Gamma}{P} \bydef{=}
    \setdefb{
      (a \cup a', g, e)
    }{
      (a, g, e) \in \Gamma, a' \subseteq P \setminus P_A
    }
  \end{equation}
%
  denotes the \emph{extension} of the interaction model $\Gamma$
  to the set of ports $P$.
\end{definition}

\todoSBin{Define equivalence $\arequiv$?}

An architecture $A$ enforces coordination constraints on the
components in $\cB$.  The interface \hl{$(V_A, P_A)$} of an
architecture $A$ contains all ports of the coordinating
components $\cC$ and some additional ports, which must belong to
the components in $\cB$.  In the application $A(\cB)$, the ports
belonging to $P_A$ can only participate in the interactions
defined by the interaction model $\Gamma$ of $A$.  Ports which do
not belong to $P_A$ are not restricted and can participate in any
interaction.  In particular, they can join the interactions in
$\Gamma$ (see \eq{im:extension}).  If the interface of the
architecture covers all ports of the system, \ie $P = P_A$, we
have $P\setminus P_A = \emptyset$ and the only interactions
allowed in $A(\cB)$ are those belonging to $\Gamma$.  Notice also
that the restrictions imposed by \defn{arch:application} on the
set of operand components $\cB$ ensure that an interaction in
$\IMextend{\Gamma}{P}$ can only refer to variables of the
participating components.
%
Finally, the definition of $\IMextend{\Gamma}{P}$ requires that
an interaction from $\Gamma$ be involved in every interaction
belonging to $\IMextend{\Gamma}{P}$.  To allow the ports from $P
\setminus P_A$ to be fired independently in $A(\cB)$, one must
have $(\emptyset, \true, \noop) \in \Gamma$.  

\begin{definition}[Composition of architectures]
  \label{defn:arch:composition}
  Let $A_i = (\cC_i, V_{A_i}, P_{A_i}, \Gamma_i)$, for $i = 1,2$,
  be two architectures.  The \emph{composition} of $A_1$ and
  $A_2$ is the architecture $A_1 \arcomp A_2 = (\cC_1 \cup \cC_2,
  V_{A_1} \cup V_{A_2}, P_{A_1} \cup P_{A_2}, \Gamma)$, where
%
  \begin{equation}
    \label{eq:arch:composition}
    \Gamma = \setdefb{
      (a, g_1 \land g_2, e_1 \expmix e_2) 
    }{
      (a \cap P_{A_i}, g_i, e_i) \in \Gamma_i,
      \text{ for } i = 1,2
    }
    \,.
  \end{equation}
\end{definition}

\begin{proposition}[Properties of $\arcomp$]
  \label{prop:arcomp:nice}
  Architecture composition $\arcomp$ is commutative and
  associative; it is idempotent if all coordinating components
  are deterministic; $A_{id} = \bigl(\emptyset, \emptyset,
  \emptyset, \{\emptyset\}\bigr)$ is its neutral element, \ie for
  any architecture $A$, we have $A \oplus A_{id} \arequiv A$.
  Furthermore, for any component $B$, we have $A_{id}(B) = B$.
\end{proposition}
%
\begin{proof}[Sketch of the proof]
  Commutativity and associativity follow from the corresponding
  properties of set union, Boolean conjunction and
  \todoSB{Check ``semi''}\hl{semilattice meet}.
%
  Suppose we have two architectures $A = A'$.  This does not
  necessarily mean that their sets of coordinating components
  coincide.  However, if all the involved coordinating components
  are deterministic, then, in any state of $(A \arcomp A')(\cB)$,
  both architectures will impose the same restrictions, enabling
  the same interactions between the coordinating and operand
  components.  Hence, we have $(A \arcomp A')(\cB) = A(\cB) =
  A'(\cB)$.  Since this holds for any set of components $\cB$, we
  conclude that $A \arcomp A' \arequiv A \arequiv A'$.
%
  The properties of $A_{id}$ follow immediately from the
  definitions of architecture application and composition.
\end{proof}


%****************************************************************
\subsection{Preservation of safety properties}
\label{secn:safety}

For a component $B$, we denote $S_{\semopen{B}}$,
$s^0_{\semopen{B}}$ the corresponding constituents of
$\semopen{B}$.

\begin{definition}[Safety properties]
  \label{defn:property}
  Let $B$ be a component.  A \emph{safety property} (below,
  simply \emph{property}) of $B$ is a state predicate $\Phi:
  S_{\semopen{B}} \rightarrow \sB$, such that $\bigl((q,
  \val{}{}) \models \Phi\bigr) \land (\val[\primeit]{}{} \order
  \val{}{})$ implies $(q, \val[\primeit]{}{}) \models \Phi$,
  where we write $(q, \val{}{}) \models \Phi$ iff $\Phi(q,
  \val{}{}) = \true$.  A property $\Phi$ is \emph{initial} if
  $s^0_{\semopen{B}} \models \Phi$.
  %% \todoSB{Do we need this?}\hl{;
  %% it is \emph{reachable} iff there exists a possibly empty path
  %% $s^0_{\semopen{B}} \goesto{a^1, \val{1}{}} s^1 \goesto{a^2,
  %%   \val{2}{}} \cdots \goesto{a^n, \val{n}{}} s^n$, such that
  %% $s^n \models \Phi$}.
\end{definition}

\todoSB{Cite the original paper somewhere.}
The main idea of our approach is that an architecture enforces
its characteristic property on the set of its operand components.
From this point of view, the set of coordinating components is
not relevant, neither are their states.  Thus, to talk about
properties enforced by architectures, we consider properties on
the unrestricted composition of the operand components as
formalized by the following definition.

\begin{definition}[Enforcing properties]
  \label{defn:impose}
  Let $A = (\cC, P_A, V_A, \Gamma)$ be an architecture; let $\cB$
  be a set of components and $\Phi$ an initial property of
  their parallel composition $A_{id}(\cB)$ (see
  \prop{arcomp:nice}).  We say that \emph{$A$ enforces $\Phi$ on
    $\cB$} iff, for every state $s = (s_c, s_b)$ reachable in
  $\semopen{A(\cB)}$, with 
  $s_c \in \prod_{C \in \cC} S_{\semopen{C}}$ and
  $s_b \in \prod_{B \in \cB} S_{\semopen{B}}$,
  we have $s_b \models \Phi$.
\end{definition}

According to the above definition, when we say that an
architecture enforces some property $\Phi$, it is implicitly
assumed that $\Phi$ is initial for the coordinated components.
Below, we omit mentioning this explicitly.

\begin{theorem}[Preserving enforced properties]
  \label{thm:combining}
  Let $\cB$ be a set of components; let $A_i = (\cC_i, V_{A_i},
  P_{A_i}, \Gamma_i)$, for $i = 1,2$, be two architectures
  enforcing on $\cB$ the properties $\Phi_1$ and $\Phi_2$
  respectively.  The composition $A_1 \arcomp A_2$ enforces on
  $\cB$ the property $\Phi_1 \land \Phi_2$.
\end{theorem}

%****************************************************************
%****************************************************************

\section{Open pNets}
\label{secn:pNets}
\todoLHin{rewrite after uniformisation}
pNets are tree-like structures, where the leaves are either
\emph{parameterised labelled transition systems (pLTSs)}, expressing the
behaviour of basic processes, or \emph{holes}, used as placeholders
for unknown processes, of which we only specify the set of possible
actions, this set is named the \emph{sort}.
Nodes of the tree (pNet nodes) are synchronising artifacts, using a
set of \emph{synchronisation vectors} that express the possible
synchronisation between the parameterised actions of a subset of the
sub-trees.

\subsection{The (open) pNets Core Model}
\label{section:pNets}


A pLTS is a labelled transition system with variables\replaceSB{; variables}{. Variables} can be
manipulated, defined, or accessed \noteSB{What does this mean?}\hl{inside states}, actions, guards, and
assignments. Without loss of generality and to simplify the formalisation, we suppose 
here that 
variables are local to each 
state: each state has its set of variables disjoint from the others. Transmitting 
variable values from one state to the other can be done by explicit assignment. 
%Similarly, to simplify the management of variables and without loss of expressivity, we 
%suppose that transitions looping to the same state does not do assignments.
Note that we make no assumption on \addSB{the }finiteness of the set of states, nor
on \addSB{the }finite branching of the transition relation.

We first define the set of actions a pLTS can use, let $a$
range over action labels\removeSB{, $\symb{op}$ are operators, and $x_i$ range over
variable names}. Action terms are\replaceSB{:}{{} of the form $\alpha=a((?x_i)^{i \in I}, 
\Expr_j^{j \in J})$, where $(?x_i)^{i \in I}$ are input variables, $\Expr_j^{j \in J})$ 
are expressions.}\noteLH{Eric suggests to remove this: too precise. I 
think last Expression should be removed not sure about the rest -- to be discussed}\noteSB{I agree: expressions are not needed, but it is nice to explain how action terms are formed }
\removeSBin{def of $\AlgA$, $p_i$, $\Expr$}
%\[
%\begin{array}[l]{rcl@{\quad}p{5.5cm}}
%  \alpha\in\AlgA&::=&a(p_1,\ldots,p_n)&\text{action terms}\\
%  p_i&::=& ?x~|~\Expr&\text{parameters (input variable or expression)}\\
%  \Expr&::=& 
%  \symb{Value}~|~x~|~\symb{op}(\Expr_1,..,\Expr_n)&\text{Expressions}
%\end{array}
%\]
\removeSB{The input variables in an action term are those marked with a
$\symb{?}$.}
We additionally suppose that each input variable does not
appear \replaceSB{somewhere else }{anywhere }in the same action term\replace[Maybe? Not sure this is necessary.\Simon]{:
$p_i=?x\Rightarrow\forall j\neq i.\, x\notin \vars(p_j)$}{.}

\begin{definition}[pLTS]
\label{pLTS}
A pLTS is a tuple
$\pLTS\triangleq\mylangle S,s_0, \to\myrangle$ where:
\begin{itemize}
\item[$\bullet$]
$S$ is a set of states\addSB{, $\vars(s)$ denotes the set of associated variables}.
\item[$\bullet$]
$s_0 \in S$ is the initial state.\noteSB{I usually put commas or semicolons in itemised lists}
%\item[$\bullet$]
 %Variables in
%$\iv(\alpha)$ are assigned by the action, other variables can be assigned
%by the additional assignments.
\item[$\bullet$] $\to \subseteq S \times L \times S$ is the transition relation and 
$L$ is the set of labels of the form
% SB: I removed a new paragraph here, ok?
$\langle \alpha,~g,~e\rangle$,
where \noteSB{I suggest using $\actions{V}$ (\ie \texttt{\textbackslash actions\{V\}}), with $V = \bigcup_{s \in S} \vars(s)$.}\hl{$\alpha \in\AlgA$} is a parameterised action, $g$ is a guard, and \hl{the variables 
$x_j\in \vars(s')$
are assigned the updates in $e$. }\noteSB{I still have to update in my section, but it seems that my initial use of expressions conflicts with that in the SVs.  I suggest writing explicitly $(?x_i := e_i)^{i \in I}$ or $\overline{?x := e}$ (not sure these should be input vars, though).}
If 
$s \xrightarrow{\langle \alpha,~g,~e\rangle} s'\in \to $ then 
% REMOVED BECAUSE USELESS: $\iv(\alpha)\!\subseteq\! \vars(s')$, 
		$\vars(\alpha)\backslash \iv(\alpha)\!\subseteq\! \vars(s)$, 
		$\vars(g)\!\subseteq\! \vars(s)\cup\vars(\alpha)$, and
		$\vars(\codom(e))\!\subseteq\! \vars(s)\cup\vars(\alpha)\land 
		\vars(\dom(e))\!\in\!\vars(s')$. %,  and $s= s'\Rightarrow J=\emptyset$. 
		
\end{itemize}
\end{definition}
\noteLH{is dom and codom clear?}
\noteSB{Not sure: I understand, because I know what is meant, but this is not a canonical use, so should be explained. Will be solved if we roll back to \mbox{$(?x_i := e_i)^{i \in I}$} or \mbox{$\overline{?x := e}$}.}
Now we define
pNet nodes, as constructors for hierarchical behavioural structures.
A pNet has a set of sub-pNets that can be either pNets or pLTSs, and a
set of \replaceSB{Holes}{holes}, playing the role of process parameters.

A composite pNet consists of a set of sub-pNets exposing
a set of actions, each of them synchronising actions in each of
the sub-pNets. The synchronisation between global actions and
internal actions is given by  \emph{synchronisation vectors}: a
synchronisation vector synchronises one or several internal actions, and
exposes a single resulting global action.
Actions involved at the pNet level (in the synchronisation vectors) do
not \replaceSB{need to distinguish  input 
variables. Action terms for pNets are defined as follows:}{have input variables, \ie they 
have the form $a(\Expr_j^{j \in J})$.}
\removeSBin{
%\[
%\begin{array}[l]{rcl@{\quad}l}
$  \alpha\in \AlgAS ::=a(\Expr_1,\ldots,\Expr_n)$
%\end{array}
%\]
}

\todoLH{Guard syntax???}
\begin{definition}[pNets]\label{def-pnets}
A pNet is a hierarchical structure where leaves are pLTSs and holes:\\\noteSB{I would not put a new line here}
$\pNet\triangleq \pLTS~|~\mylangle \pNet_i^{i\in I}, J, \symb{SV}_k^{k\in 
K}\myrangle$
where
\begin{itemize}
\item[$\bullet$] \noteSB{This is clear from $\pNet_i^{i\in I}$.}\hl{$I \in \I$ is the set over which sub-pNets are indexed.}
\item[$\bullet$] $\pNet_i^{i\in I}$ is the family of sub-pNets.
%  $\pNet_i^{i\in I}$ is a family of sub-pNets where $I\in\I_\P$ is the set over which 
%sub-pNets are indexed.

\item[$\bullet$] \noteSB{$I_P$ is not defined and, probably, not necessary.}\hl{$J\!\in\!\I_\P$} is a set of indexes, called \emph{holes}.
$I$ and $J$ are \emph{disjoint}: $I\!\cap\! J=\emptyset$,  $I\!\cup\! J\neq\emptyset$
%\item[$\bullet$] $\Sort_j \subseteq \AlgAS$ is a set of action terms, denoting the 
%\emph{sort} of
%hole $j$.

\item[$\bullet$] $\symb{SV}_k^{k\in K}$ is a set of
  synchronisation vectors (\hl{$K\in\I_\P$}). $\forall k\!\in\! K,
  \symb{SV}_k\!=\!\alpha_{l}^{l\in I_k \uplus J_k}\to\alpha'_k\,|\,g_k$, where
  $\alpha'_k\in \mathcal{A}_\P$, $I_k\subseteq I$, $J_k\subseteq J$,
  $\forall i\!\in\!
  I_k.\,\alpha_{i}\!\in\!\Sort(\pNet_i)$,  $\forall j\!\in\!
  J_k.\,\alpha_{j}\!\in\!\Sort_j$, and $\vars(\alpha'_k)\subseteq \bigcup_{l\in I_k\uplus 
  J_k}{\vars({\alpha_l})}$. The global action of a vector $\symb{SV}_k$ is
\replaceSB{$\Label(\symb{SV}_k) = \alpha'_k$}{$\alpha'_k$, also denoted $\Label(\symb{SV}_k)$}. \replaceSB{$g_k $ is a guard associated to the vector 
$\vars(g_k)\subseteq \bigcup_{l\in I_k\uplus J_k}{\vars({\alpha_l})}$.}{The Boolean expression $g_k $, such that $\vars(g_k)\subseteq \bigcup_{l\in I_k\uplus J_k}{\vars({\alpha_l})}$, is a guard associated to the vector.}


\end{itemize}
\end{definition}

The preceding definition relies on the auxiliary functions below:

\begin{definition}[Sorts, Holes, Leaves of pNets]
  \begin{itemize}
  \item  \noteSB{I think this can be lightened up a bit, since the definitions are all straightforward, whereas the notation is complex.  Maybe reformulate in words and move the formal definitions in the appendix?}
 The sort of a pNet is its signature, i.e. the set of actions it can
perform. \noteSB{Is there any chance that, by writing assignments explicitly ($x := e$), we could eliminate the need to distinguish the input variables at all? That would simplify things.}\hl{In the definition of sorts, we do not need to distinguish
input variables} (that specify the dataflow within LTSs), so for
computing LTS sorts, we use a substitution operator\footnote{$\subst{y_k\gets x_k}^{k\in 
K}$ is the parallel substitution 
operation.} to remove the
\emph{input marker} of variables. Formally:
\[
\begin{array}{l}
\Sortop(\mylangle S,s_0, \to\myrangle) = \{\alpha\subst{x \gets ?x| 
x\in\symb{iv}(\alpha)}|s \xrightarrow{\langle \alpha,~e_b,~(x_j\!:= {e}_j)^{j\in
    J}\rangle} s'\in \to \} \\
\Sortop(\mylangle \overline{\pNet}\!, %\pNet_i^{i\in I}, \Sort_j^{j\in J}
\overline{\pNet}\!,
\overline{\symb{SV}}\myrangle)
=\{\alpha' |\, \overline{\alpha}%\alpha_j^{j\in J_k}
\to\alpha'\in\set{\symb{SV}}\}
\end{array}
\]

\item
The set of holes of a pNet is defined inductively; the sets of holes
in a pNet node and its subnets are all disjoint:
  \[\begin{array}{l}
\Holes(\mylangle S,s_0, \to\myrangle) \!=\! \emptyset \\
\Holes(\mylangle \pNet_i^{i\in I}\!,\Sort_j^{j\in J}\!, \overline{\symb{SV}}\myrangle) 
=J\uplus{\displaystyle \bigcup_{i\in 
I}\Holes(\pNet_i)}\\
\forall i\in I.\, \Holes(\pNet_i)\cap J=\emptyset\\
\forall i_1,i_2\in I.\,i_1\neq i_2\Rightarrow  
\Holes(\pNet_{i_1})\cap\Holes(\pNet_{i_2})=\emptyset
\end{array}\]
\item
The set of leaves of a pNet is the set of all pLTSs occurring in the structure, defined 
inductively as:
\[\begin{array}{l}
\Leaves(\mylangle S,s_0, \to\myrangle) \!=\!\{\mylangle S,s_0, \to\myrangle\}\\
\Leaves(\mylangle \pNet_i^{i\in I}\!,%Sort_j^{j\in J}
\overline{\Sort}\!, \overline{\symb{SV}}\myrangle) = {\displaystyle \biguplus_{i\in 
I}\Leaves(\pNet_i)\uplus\{i\mapsto \pNet_i|\pNet_i\pNet_i \text{ is a }\pLTS\}}
\end{array}\]
\end{itemize}
\end{definition}

\noteLHin{Below a simpler def 11}
\begin{definition}[Sorts, Holes, Leaves of pNets]
  \begin{itemize}
  \item 
 The sort of a pNet is its signature, i.e. the set of actions it can
perform. A sort is a set of action signature, where each action signature is an action 
label plus the arity of the action.

\item
The set of holes $\Holes(\pNet_{i_1})$ of a pNet is the indexes of the holes of the pNet 
itself plus the indexes of all the holes of its subnets (we suppose those indexes 
disjoints).
%
%defined inductively; the sets of holes
%in a pNet node and its subnets are all disjoint:
%  \[\begin{array}{l}
%\Holes(\mylangle S,s_0, \to\myrangle) \!=\! \emptyset \\
%\Holes(\mylangle \pNet_i^{i\in I}\!,\Sort_j^{j\in J}\!, \overline{\symb{SV}}\myrangle) 
%=J\uplus{\displaystyle \bigcup_{i\in 
%I}\Holes(\pNet_i)}\\
%\forall i\in I.\, \Holes(\pNet_i)\cap J=\emptyset\\
%\forall i_1,i_2\in I.\,i_1\neq i_2\Rightarrow  
%\Holes(\pNet_{i_1})\cap\Holes(\pNet_{i_2})=\emptyset
%\end{array}\]
\item
The set of leaves of a pNet is the set of all pLTSs occurring in the structure, as an 
indexed family of the form $\Leaves(\pNet)= \mylangle \pNet_i \myrangle^{i \in L}$.
is said to be \emph{open}.
\end{itemize}
\end{definition}
\subsection{Operational Semantics for Open pNets}
\label{section:op-semantics}

\noteLH{a pred is a g a post is a e}
The semantics of open pNets will be defined  as an open automaton. An open
automaton is an automaton where \noteSB{This might give a false impression that there is some kind of CCS-like ``duality'' between pLTSs and holes}\hl{each transition composes transitions of several LTSs with
actions of some holes}, the transition occurs if some predicates hold, and can involve a 
set of state modifications.
%\TODO{adopt a uniform notation for open transitions, almost each instance has a 
%different 
%notation! I suggest p,l,pr,po using \{\} for p,l,po as they are sets}
\begin{definition}[Open transitions]
	\label{def:OpenTransitions}
	An \emph{open transition} over a set \noteSB{In the architecture section, I say that we omit indices on $\goesto{}$ and that they are always clear from the context.}\hl{$(S_i,s_{0 i}, \rightarrow_i)^{i\in
	I}$} of LTSs, a
	set $J$ of holes with sorts $\Sort_j^{j\in J}$, and a set of states $\mathcal{S}$ is 
	a structure of the form:	
	\begin{mathpar}
	\inferrule*[myfraction=\reddottedrule]
	{\{s_i~{\xrightarrow{a_i}}_i ~s_i^{\prime}\}^{i\in I},
		\{\xrightarrow{b_j}_j\}^{j\in J}, \Pred, \Post}
	{s \OTarrow {v}s'}
	\end{mathpar}
	Where $s, s'\in\mathcal{S}$ and for all
        $i\in I$, $s_i{\xrightarrow{a_i}}_i s_i^{\prime}$ is a transition of the
	LTS $(S_i,s_{0 i}, \rightarrow_i)$, and \noteSB{Why insist on ``transition'' and $\goesto{b_j}_j$? Why not simply ``$b_j$ is an action in the sort $\Sort_j$?}\hl{$\xrightarrow{b_j}_j$
        is a transition of the hole $j$}, for any action $b_j$ in the
        sort $\Sort_j$.  \noteSB{Why $v$? Why not $\alpha$? As now, it is not clear whether this is an action term or a variable $v \in \vars(s)$ of some hypothetical state $s$.}\hl{$v$ is a variable denoting the resulting action}
        of this open transition. \Pred\ is a predicate 
	over the different variables of the
	terms, labels, and states $s_i$, $b_j$, $s$, $v$. \Post\ is a set of equations that 
	hold \emph{after the open transition}, they are represented as a substitution of the 
	form $\{x_k\gets \Expr_k\}^{k\in K}$ \noteLH{considering the level of abstraction I 
	would like to just put $\Post_=e$}
	where $x_k$ are variables of $s'$, $s'_i$, and $e_k$ are expressions over the other 
	variables of the open transition.
\end{definition}


\begin{example}\emph{An open-transition.}
  \label{OT:enable-composed}
  The \texttt{EnableCompL} pNet of Fig. \ref{schema:enable-composed} has 2 controllers 
  and 2 holes. One of its possible open-transition is:

 \smallskip
 $  OT_2  = \openrule{
                       0 \xrightarrow{\delta}_{C_3} 1 ~~
                            0 \xrightarrow{l}_{C_4} 0  ~~                 
                            \xrightarrow{\delta(x4)}_P ~~
                            \xrightarrow{accept(x4)}_Q 
                      }
  %  {\ostate{00} \xrightarrow{\underline{\delta(x1)}} \ostate{10}}
    {A1_0 \OTarrow{\underline{\delta(x4)}} A1_1}
  $\noteSB{The underline notation did not appear until here.}
\end{example}


\begin{definition}[Open automaton]
	\label{def:open-automaton}
	An \emph{open automaton} is a structure\\ $A =
	\langle\LTS_i^{i\in I},J,\mathcal{S},s_0,\mathcal{T}\rangle$, where:
	\begin{itemize}
		\item[$\bullet$]  $I$ and $J$ are  sets of indices,
		\item[$\bullet$]  $\LTS_i^{i\in I}$ is a family of LTSs,
		\item[$\bullet$]   $\mathcal{S}$ is a set of states and $s_0$ an initial state
		among $\mathcal{S}$,
		\item[$\bullet$] $\mathcal{T}$ is a set of open transitions and for each
		$t\in \mathcal{T}$ there exist $I'$, $J'$ with $I'\subseteq I$, $J'
		\subseteq J$, such that $t$ is an open transition over $\LTS_i^{i\in I'}$, $J'$,
		and  $\mathcal{S}$.
		
	\end{itemize}
\end{definition}
	

%
%Then the semantics of a pNet is characterized by a set of {\em open
%transitions}, where the hypotheses on process parameters are
%replaced by 1) transitions of the pLTSs at the leaves, and 2) formal
%hypotheses on the transitions of the holes. A {\em predicate} is used
%to relate the parameters and names appearing in the actions of the
%leaves and the holes involved in the rules, but also appearing in  the resulting action.

We define states of open pNets as tuples of states, where we denote tuples
in structured states as $\triangleleft\ldots\triangleright$ for distinguishing tuple 
states from other tuples.
\begin{definition}[States of open pNets]\label{def-states}
  A state of an open pNet is a tuple (not necessarily finite) of the
  states of its leaves.

  For any pNet p, let $\Leaves(p) = \mylangle S_i,{s_i}_0, \to_i\myrangle^{i \in L}$ be 
  the set of pLTS at its leaves,
  then $States(p) = \{\triangleleft s_i^{i\in L}
  \triangleright| \forall i\in L. s_i \in S_i\}$.
A pLTS being its own single leave:
  $States(\mylangle S,s_0, \to\myrangle) = \{\triangleleft s \triangleright| s \in S\}$.

The initial state is defined as:
$InitState(p) = \triangleleft {{s_i}_0}^{i\in L}  \triangleright$.
\end{definition}



%% \begin{example} \emph{State of a pNet}
%%   The states of pNet \texttt{EnableCompL} are:
%%   $\triangleleft 00 \triangleright, \triangleleft 10 \triangleright, \triangleleft 11 
%%\triangleright$
%% \end{example}

\paragraph{Predicates:}
Let
$\mylangle\overline{\pNet},\overline{\Sort},\overline{\symb{SV}}\myrangle$
be a pNet. Consider a synchronisation vector $\symb{SV}\in \overline{\symb{SV}}$. We 
define a
predicate $\Pred$ relating
the actions of the involved sub-pNets and the resulting actions. This predicate verifies:
\begin{multline}
  \Pred_{\symb{sv}}(SV, 
  a_i^{i\in I}, b_j^{j\in J}, v)\Leftrightarrow
  \\
%\exists 
%%   \bigg(
%% \begin{array}{l}
\exists {\vars((a'_i)}^{i\in I}),
{\vars((b'_j)}^{j\in J}),\vars(v').\, %SV=
%\\
%\land
\\\forall i\in I.\, a_i=a'_i\land \forall j \in J.\, b_j=b'_j \land v=v' \land g
%% \end{array} \bigg)
\\
\text{ Where } SV=\bigl({(a'_i)}^{i\in I}, {(b'_j)}^{j\in J}\rightarrow v' | g\bigr)
\end{multline}
\noteLH{Is and g sufficient? I would say yes}
\noteSB{In principle, yes. I am not 100\% sure that the implicit dependency of $g$ on the variables in $a'_i$ and $b'_j$ is clear.}

Somehow, this predicate entails a verification of satisfiability in the sense that if the 
predicate $\Pred_{\symb{sv}}$ is not satisfiable, then the transition associated with the 
synchronisation will not occur in the considered state. 


In any other case (if the action families do not match or if there is no valuation of
variables such that the above formula can be ensured) the predicate is undefined.

This definition is not constructive but it is easy to build the predicate constructively
by brute-force unification of the sub-pNets
actions with the corresponding vector actions, possibly followed by a simplification
step.


We build the semantics of open pNets as an open automaton where LTSs are the pLTSs at
the leaves of the pNet structure, and the states are given by 
Definition~\ref{def-states}. The open transitions first
 project the global state into states of the leaves, then apply
pLTS transitions on these states, and compose them with the sort of the holes. %The pNet
%structure does not appear in the open-automaton, only the
%set of Holes and the set of Leaves.
The semantics   regularly instantiates \emph{fresh} variables, and uses a
\emph{clone} operator that clones a term replacing each variable with a
fresh one.


\begin{definition}[Operational semantics of open pNets]
	\label{def:operationalSemantics}
	The semantics of a pNet $p$ is an open automaton $A = 
	\langle\symb{Leaves}(p),\symb{Holes}(p),\symb{States}(p),\symb{InitState}(p),
	\mathcal{T}\rangle$, where \noteSB{We use $\cT$ for term algebra}\hl{$\mathcal{T}$} is the smallest set of open transitions		
	satisfying the rules below:
%	\begin{itemize}
%		\item $J$ is the set of holes: $Holes(p)= J$. 
		%  \item $\overline{L}^L = Leaves(p), \overline{H}^J = Holes(p)$
%		\item ${\mathcal{S}} = States(p)$ and $s_0 = InitState(p)$
%		\item $\mathcal{T}$ is the smallest set of open transitions		satisfying the 
%rules below:
%	\end{itemize}
	
	%% \TODO{ We should be careful here: after (re) reading "Huimin
	%% 	Lin, 'Symbolic Transition Systems with Assignements', Concur'96" I
	%% 	think handling assignments is not trivial, even for comparisons of pLTSs. }


	
	The rule for a pLTS  checks that the guard 
	is verified and transforms assignments into post-conditions:
	
	\begin{description}
		\item[{\bf Tr1:}]
		$\inferrule
		{ s \xrightarrow{\langle \alpha,~g,~e\rangle} s'\in \to  }
		{ \mylangle  S,s_0, \to \myrangle
			\models
			\inferrule*[myfraction=\reddottedrule]
			{\{s \xrightarrow{\alpha} s'\} ,\emptyset ,
			g,e}
			{\ostate{s} \OTarrow{\alpha} \ostate{s'}}
		}$
	\end{description}
%	Note that this note is greatly simplified by the fact that variables are local to 
%	thread; introducing global state variables or accepting loops to the same 
%	state would 
%	require to reason 
%	on the scope of 
%	each variables, and to introduce additional variables to handle the several occurence 
%	of the same pLTS variable in the predicates. Indeed the constraints on pLTS 
%	transitions 
%	ensure that the same variable never appears both on the left and on the right of the 
%	equations of a predicate.
	
	The second rule deals with pNet nodes: for each possible
	synchronisation vector applicable to the rule subject, the premises
	include one {\em open transition} for each sub-pNet involved, one possible
	{\em action} for each Hole involved, and the predicate relating these
	with the resulting action of the vector.
	A key to understand this rule is that the open transitions are
	expressed in terms of the leaves and holes of the pNet structure,
	i.e. a \noteSB{Is this compatible with the ``compositionality'' argument?}\hl{flattened view of the pNet}: e.g. $L$ is the index set of the
	Leaves, $L_m$ the index set of the leaves of one subnet indexed $m$, so all $L_m$
	are disjoint subsets of $L$. Thus the states in the open transitions,
	at each level, are tuples including states of all the
	leaves of the pNet, not only those involved in the chosen
	synchronisation vector.
	\begin{description}
		\item[{\bf Tr2:}]\noteSB{I do not understand why the second line $pNet_m \models\dots$ is not subsumed by the first one?}
	\end{description}
	
	\noindent
    $\inferrule
    {k\!\in\! K \\SV_k \!=\! \alpha_m^{m \in I\uplus J} \!\to\! 
    \alpha' |g \\
    	Leaves(\mylangle \pNet_i^{i\in I'}, \overline{\Sort}, \symb{SV}_k^{k\in 
    	K}\myrangle) \!=\! \pLTS_l^{l\in L} \\    	
    	\forall m\!\!\in\!\! I. 	
    {\left(\begin{array}{l}
	{\pNet_m \models\inferrule*[myfraction=\reddottedrule]
    	{\{s_{i}\xrightarrow{a_{i}}_i s_{i}'\}^{i\in I_m^\prime},
    	\{\xrightarrow{b_{j}}_j\}^{j\in J'_m}, \Pred_m, \Post_m}
    	{\ostate{s_{i}^{i \in L_m}} \OTarrow {v_m}
    		\ostate{(s_i^\prime)^{i\in L_m}}}\lor }\\
		{ \pNet_m 
    	 \models
    	\inferrule*[myfraction=\reddottedrule]
    	{\{s_m \xrightarrow{v_m} s_m'\},\emptyset, \Pred_m, \Post_m}
    	{\ostate{s_m} \OTarrow {v_m}
    		\ostate{s_m'}} \land I'_m=\{m\} \land J'_m=\emptyset}
\end{array}\right)}\\
    	%\land
    	%Leaves(\pNet_m) = \overline{\pLTS}^{L_k})  	
     J' = \biguplus_{m\in I}\!\! J'_m \uplus J 	\\
    	\Pred = \bigwedge_{m\in I}\!\! \Pred_m \land
    	\Pred_{\symb{sv}}(SV_k,v_m^{m\in I},b_j^{j\in J},v)\\ 
    		I' = \biguplus_{m\in I}\!\! I_m'
    	\\\forall i\in	L\backslash I'.\,s'_i=s_i \\
    {\tt fresh}(\alpha_m,\alpha',b_j,v) 
    }
    {\mylangle \pNet_i^{i\in I'}, \overline{\Sort}, \symb{SV}_k^{k\in K}\myrangle
    	\models
    	{\inferrule*[myfraction=\reddottedrule]
    		{\{s_i\xrightarrow{a_i}_i s_i^{\prime}\}^{i\in I^\prime},
    		\{\xrightarrow{b_j}_j\}^{j\in J^\prime}, \Pred, \uplus_{m\in I_k} 
    		\Post_m}
    		{\ostate{s_i^{i\in L}} \OTarrow {v}
    			\ostate{(s_i^\prime)^{i\in L}}}
    	}
    }
    $
	\medskip
%        \TODO{may be explain how $\Pred(SV,a_i^{i\in I_k},b_j^{j\in
%            J_k},v)$ is built ? You mean more than what is written on previous page????}
	%%    \TODO{I have tentatively added the sort constraint on hole actions, that was
	%%not included in the first version... I'm unsure whether this is the best place to
	%%include it, because it may change the decidability conditions on predicates}
	
\end{definition}


%****************************************************************
%****************************************************************

\section{Encoding of architectures into open pNets}
\label{secn:encoding}

%****************************************************************
%****************************************************************

\section{SMT encoding of open pNets}
\label{secn:smt}

%****************************************************************
%****************************************************************

\section{Case study}
\label{secn:case-study}

%****************************************************************
%****************************************************************

\section{Related work}
\label{secn:related}

%****************************************************************
%****************************************************************

\section{Conclusion}
\label{secn:conclusion}

%****************************************************************
%****************************************************************

\bibliographystyle{abbrv}
\bibliography{biblio.bib}

%****************************************************************
%****************************************************************
\appendix
\clearpage

\section{Proofs}

\begin{lemma}
  \label{lem:onlyone}
  Let $A = (\cC, V_A, P_A, \Gamma)$ be an architecture and denote
  by $\Gamma_\cC \bydef{=}
%
  \setdef{
    (a \cap P_\cC, \true, \noop)
  }{
    (a, g, e) \in \Gamma
  }$, with $P_\cC = \bigcup_{C \in \cC} P_C$,
%  
  the projection of $\Gamma$ onto the coordinating components of
  $A$.  Consider the architecture $A' = (\{C'\}, V_A, P_A,
  \Gamma)$, where $C' = \Gamma_\cC(\cC)$.  For any set of
  components $\cB$, satisfying the conditions of
  \defn{arch:application}, we have
  $\semopen{A(\cB)} = \semopen{A'(\cB)}$.
\end{lemma}
%
\begin{proof}
  First of all, notice that, by \defn{im},
  $V_{C'} = \bigcup_{C \in \cC} V_C$ and $P_{C'} = P_\cC$.
  The conditions of \defn{arch} are satisfied and $A'$ is indeed
  an architecture.  Furthermore, $\cB$ satisfies the conditions
  of \defn{arch:application} \wrt $A'$.  Hence, the component
  $A'(\cB)$ is well defined.

  Clearly the state spaces, initial states and interfaces of both
  components coincide.  Thus we only have to prove that so do the
  transition relations.  Let us assume that
  $\cC = C_i^{i \in I}$ and $\cB = B_j^{j \in J}$.
  We will use $q_i, q_i'$ to denote the states of
  $C_i$ and $q_j, q_j'$ to denote the states of $B_j$,
  and similarly for the valuations of variables.

  By \defn{comp:semantics}, a transition
%
  \begin{equation}
    \label{eq:lem1:trans:sem}
    (q_i, \val{}{i})^{i \in I} (q_j, \val{}{j})^{j \in J}
    \goesto{a, \val[\tilde]{}{}}
    (q_i', \val{\prime}{i})^{i \in I}
    (q_j', \val{\prime}{j})^{j \in J}
  \end{equation}
%  
  is a transition in $\semopen{A(\cB)}$ (\resp
  $\semopen{A'(\cB)}$) iff
%
  \begin{equation}
    \label{eq:lem1:trans}
    q_i^{i \in I} q_j^{j \in J}
    \goesto{a, G, E}
    (q_i')^{i \in I} (q_j')^{j \in J}
    \,,
  \end{equation}
%
  is a transition in $A(\cB)$ (\resp $A'(\cB)$) and
%
  \begin{mathpar}
    \val{i \in I}{i} \val{j \in J}{j}
    \models G
    \,,
    \and
    (\val{\prime}{i})^{i \in I} (\val{\prime}{j})^{j \in J}
    = \val[\tilde]{}{}[E]
    \,,
    \and
    \valdiff{\val{i \in I}{i} \val{j \in J}{j}}{
      \val[\tilde]{}{}} \subseteq \export(a)
    \,.   
  \end{mathpar}

  By \defn{im}, \eq{lem1:trans} is a transition in $A(\cB)$
  iff $a \neq \emptyset$, $(a, g, e) \in \IMextend{\Gamma}{P}$,
  where $P = \bigcup_{B \in \cB \cup \cC} P_B$, and
  %
  \begin{enumerate}
  \item for $i \in I$,
    $q_i \goesto{a \cap P_{C_i}, g_{C_i}, e_{C_i}} q_i'$
    is a transition in $C_i$, or
    $a \cap P_{C_i} = \emptyset$ and $q_i = q_i'$,
    $g_{C_i} = \true$ and $e_{C_i} = \noop$\,;
  \item for $j \in J$,
    $q_j \goesto{a \cap P_{B_j}, g_{B_j}, e_{B_j}} q_j'$
    is a transition in $B_j$, or
    $a \cap P_{B_j} = \emptyset$ and $q_j = q_j'$,
    $g_{B_j} = \true$ and $e_{B_j} = \noop$\,;
  \item $G = g \land \bigwedge_{i \in I} g_{C_i} \land
    \bigwedge_{j \in J} g_{B_j}$ and
    $E = e; e_{C_i}^{i \in I}, e_{B_j}^{j \in J}$.
  \end{enumerate}
  %

  Similarly, \eq{lem1:trans} is a transition in
  $A'(\cB)$ iff $a \neq \emptyset$,
  $(a, g, e) \in \IMextend{\Gamma}{P}$ and
  %
  \begin{enumerate}
  \item \label{only1:coord}
    $(q_i)^{i \in I}
    \goesto{a \cap P_{C'}, g_{C'}, e_{C'}}
    (q_i')^{i \in I}$ is a transition in $C'$,
    or $a \cap P_{C'} = \emptyset$ and $q_i = q_i'$,
    for all $i \in [1,m]$,
    $g_{C'} = \true$ and $e_{C'} = \noop$\,;
  \item for $j \in J$,
    $q_j \goesto{a \cap P_{B_j}, g_{B_j}, e_{B_j}} q_j'$
    is a transition in $B_j$, or
    $a \cap P_{B_j} = \emptyset$ and $q_j = q_j'$,
    $g_{B_j} = \true$, and $e_{B_j} = \noop$;
  \item $G = g \land g_{C'} \land
    \bigwedge_{j \in J} g_{B_j}$ and
    $E = e; e_{C'}, e_{B_j}^{j \in J}$.
%
  \breakenumistart
  Notice that, for any $i \in I$, since
  $P_{C_i} \subseteq P_{C'}$, we have
  $a \cap P_{C_i} = (a \cap P_{C'}) \cap P_{C_i}$.
  Thus, if $a \cap P_{C'} \neq \emptyset$, 
  the transition in condition~\ref{only1:coord} above is
  present in $C'$ iff
  $(a \cap P_{C'}, \true, \noop) \in \Gamma_\cC$ and
  \breakenumiend
% 
  \item for $i \in I$,
    $q_i \goesto{a \cap P_{C_i}, g_{C_i}, e_{C_i}} q_i'$
    is a transition in $C_i$, or $a \cap P_{C_i} = \emptyset$
    and $q_i = q_i'$,
    $g_{C_i} = \true$ and $e_{C_i} = \noop$,
  \item $g_{C'} = \bigwedge_{i \in I} g_{C_i}$ and
    $e_{C'} = e_{C_i}^{i \in I}$.
  \end{enumerate}

  Consider $(a,g,e) \in \IMextend{\Gamma}{P}$.  By
  \eq{im:extension}, this is equivalent to
  $(a \cap P_A, g, e) \in \Gamma$.
  Since $P_{C'} = P_\cC \subseteq P_A$, we have
  $a \cap P_{C'} = a \cap P_\cC = (a \cap P_A) \cap P_\cC$.
  Hence, this is also equivalent to
  $(a \cap P_{C'}, \true, \noop) \in \Gamma_\cC$,
  which concludes the proof.
\end{proof}

\begin{lemma}
  \label{lem:onestep}
  Let $\cB$ be a set of components; let $A_i = (\{C_i\}, V_{A_i},
  P_{A_i}, \Gamma_i)$, for $i = 1,2$, be two architectures
  applicable to $\cB$, with one coordinator each, and denote
  $P_i = P_{C_i} \cup \bigcup_{B \in \cB} P_B$.
  Finally, let
%
  $
  s_1 s_2 s
%
  \goesto{a, \tilde{\val{}{}}}
%
  s_1' s_2' s'
  $,
%
  be a transition in $\semopen{(A_1 \arcomp A_2)(\cB)}$, with
  $a \neq \emptyset$, 
  $s, s' \in \prod_{B \in \cB} S_{\semopen{B}}$
  and
  $s_i, s_i' \in S_{\semopen{C_i}}$ ($i=1,2$).
%
  Then, for $i=1,2$, either $a \cap P_i = \emptyset$ and $s_i s=
  s_i' s'$, or there exists a transition
%
  $
  s_i s
%
  \goesto{a \cap P_i, \val[\doubletilde]{}{}}
%
  s_i' s''
  $ 
%
  in $\semopen{A_i(\cB)}$, with
  $\val[\doubletilde]{}{}$ being the restriction of
  $\val[\tilde]{}{}$ to
  $V_1 = V_{C_1} \cup \bigcup_{B \in \cB} V_B$ and
  $s' \order s''$.
\end{lemma}
%
\begin{proof}
  By \defn{comp:semantics}, every semantic state consists of a
  component state and a valuation of component variables.  Thus,
  we have
%
  \[
  (q_1, \val{}{1}, q_2, \val{}{2}, q, \val{}{})
%
  \goesto{a, \tilde{\val{}{}}}
%
  (q_1', \val[\primeit]{}{1}, q_2', \val[\primeit]{}{1}, q', \val[\primeit]{}{})
  \,.
  \]
%
  By \eq{comp:semantics} and \eq{im:int}, this implies
  $(q_1, q_2, q) \goesto {a, G, E} (q_1', q_2', q')$, with
%
  $G = g \land g_1 \land g_2 \land g_b$
  and
  $E = e; (e_1, e_2, e_b)$,
  where
%    
  \begin{mathpar}
    g_b = \bigwedge_{B \in \supp{a} \cap \cB} g_B
    \,,
    \and
    e_b = e_B^{B \in \supp{a} \cap \cB}
  \end{mathpar}
%
  and such that $(a, g, e) \in \Gamma$ (see
  \eq{arch:composition}) and
%
  \begin{align}
    \label{eq:guard}
    (\val{}{1}, \val{}{2}, \val{}{}) &\models G
    \,,
    \\
    \label{eq:substitution}
    (\val[\primeit]{}{1}, \val[\primeit]{}{2}, \val[\primeit]{}{})
    &= \val[\tilde]{}{}[E]
    \,,
    \\
    \label{eq:transient}
    \valdiff{(\val{}{1}, \val{}{2}, \val{}{})}{\val[\tilde]{}{}}
    &\subseteq \export(a)
    \,.
  \end{align}
%
  If $a \cap P_1 = \emptyset$, then $\supp{a} = \{C_2\}$ and, by
  \eq{transient} and \defn{im}, $s_1 = s_1'$ and $s = s'$
  (recall, \defn{im}, that, for $(a,g,e) \in \Gamma$, we have $e
  \in \exprs{V_a}$).  Hereafter, assume $a \cap P_1 \neq
  \emptyset$.

  Denoting $P_b = \bigcup_{B \in \supp{a} \cap \cB} P_B$, we
  have, by \eq{im:int},
%
  \begin{mathpar}
    q_1 \goesto{a \cap P_{C_1}, g_1, e_1} q_1'
    \and
    \text{and}
    \and
    q \goesto{a \cap P_b, g_b, e_b} q'
    \,.
  \end{mathpar}
 
  By \eq{arch:composition}, for $i=1,2$, there exist
  $(a \cap P_{A_i}, g^i, e^i) \in \Gamma_i$,
  such that $g = g^1 \land g^2$ and $e = e^1 \land e^2$.
  By \eq{guard}, we have
  $(\val{}{1}, \val{}{}) \models g^1 \land g_1 \land g_b$.

  Denoting $\val[\doubletilde]{}{}$ the restriction of
%
  $\val[\tilde]{}{}$ to
  $V_1 = V_{C_1} \cup \bigcup_{B \in \cB} V_B$,
%
  we have, by \eq{substitution} and the monotonicity of
  expressions,
%
  $(\val[\primeit]{}{1}, \val[\primeit]{}{}) \order
  \val[\doubletilde]{}{}[e^1; (e_1, e_b)]$.  Notice that all
  variables affected by both $e^1$ and $e^2$ must necessarily
  belong to $\bigcup_{B \in \cB} V_B$.  Hence,
%
  $\val[\doubletilde]{}{}[e^1; (e_1, e_b)] =
  (\val[\primeit]{}{1}, \val[\doubleprimeit]{}{})$ with 
  $\val[\primeit]{}{} \order \val[\doubleprimeit]{}{}$.

  By \eq{comp:semantics} and \eq{im:int}, we have
  \[
  (q_1, \val{}{1}, q, \val{}{})
  \goesto{a \cap P_1, \val[\doubletilde]{}{}}
  (q_1', \val[\primeit]{}{1}, q, \val[\doubleprimeit]{}{})
  \,.
  \]
\end{proof}

\begin{lemma}
  \label{lem:stepabove}
  Consider a component $B$ and let 
%
  $
  (q_1, \val{}{1})
%
  \goesto{a, \val[\tilde]{}{}}
%
  (q_2, \val{}{2})
  $,
%
  be a transition in $\semopen{B}$.  For any valuation
  $\val[\primeit]{}{1}$, such that
  $\val{}{1} \order \val[\primeit]{}{1}$ and
  $\valdiff{\val{}{1}}{\val[\primeit]{}{1}} \subseteq \export(a)$,
  there exists a transition
%
  $
  (q_1, \val[\primeit]{}{1})
%
  \goesto{a, \val[\tilde]{}{}}
%
  (q_2, \val{}{2})
  $
%
  in $\semopen{B}$.
\end{lemma}
%
\begin{proof}
  By \eq{comp:semantics}, we have
%
  \begin{mathpar}
    q_1 \goesto{a, g, e} q_2
    \,,
    \and
    \val{}{1} \models g
    \,,
    \and
    \val{}{2} = \val[\tilde]{}{}[e]
    \,,
    \and
    \text{and}
    \and
    \valdiff{\val{}{1}}{\val[\tilde]{}{}} \subseteq \export(a)
    \,.
  \end{mathpar}

  By monotonicity of guards, we deduce
  $\val[\primeit]{}{1} \models g$.
  For any $v \in \export(a)$,
  $\val[\primeit]{}{1}(v) = \val{}{1}(v) = \val[\tilde]{}{}(v)$.
  Hence, 
  $\valdiff{\val[\primeit]{}{1}}{\val[\tilde]{}{}}
  \subseteq \export(a)$.
  By \eq{comp:semantics}, we conclude 
%
  $
  (q_1, \val[\primeit]{}{1})
%
  \goesto{a, \val[\tilde]{}{}}
%
  (q_2, \val{}{2})
  $.  
\end{proof}

\begin{proof}[Proof of \thm{combining}]
  By \lem{onlyone}, we can assume that each of the two
  architectures has only one coordinating component.  For $i =
  1,2$, we denote $\{C_i\} = \cC_i$, $P_i = P_{C_i}
  \cup\ \bigcup_{B \in \cB} P_B$ and $V_i = V_{C_i}
  \cup\ \bigcup_{B \in \cB} V_B$.
  
  The initiality of $\Phi_1 \land \Phi_2$, is trivial: both
  $\Phi_1$ and $\Phi_2$ are initial, hence $s^0 \models \Phi_1
  \land \Phi_2$.

  Consider a path
%
  \[
  s^0_1 s^0_2 s^0
%
  \goesto{a^1, \val[\tilde]{1}{}}
%
  s^1_1 s^1_2 s^1
%
  \goesto{a^2, \val[\tilde]{2}{}}
  \cdots
  \goesto{a^k, \val[\tilde]{k}{}}
%
  s^k_1 s^k_2 s^k
  \]
%
  in $\semopen{(A_1 \arcomp A_2)(\cB)}$, where
  $s^0,\dots,s^k \in \prod_{B \in \cB} S_{\semopen{B}}$ and
  $s^0_i,\dots, s^k_i \in S_{\semopen{C_i}}$, for $i=1,2$.
  We have to show that $s^k \models \Phi_1 \land \Phi_2$.

  Assuming that $a^1 \cap P_1 \neq \emptyset$, by \lem{onestep},
  there exists $s^{1\prime} \in \prod_{B \in \cB} S_{\semopen{B}}$,
  such that
%
  \[
  s^0_1 s^0
%
  \goesto{a^1 \cap P_1, \val[\doubletilde]{1}{}}
%
  s^1_1 s^{1\prime}
  \]
%
  is a transition in $\semopen{A_1(\cB)}$ with
  $\val[\doubletilde]{1}{}$ being the restriction of
  $\val[\tilde]{1}{}$ to $V_1$ and
  $s^1 \order s^{1\prime}$.
%
  By \lem{stepabove}, 
  \[
  s^1_1 s^1_2 s^{1\prime}
%
  \goesto{a^2, \val[\tilde]{2}{}}
%
  s^2_1 s^2_2 s^2
  \]
  is a transition in $\semopen{(A_1 \arcomp A_2)(\cB)}$ and,
  therefore, 
  \[
  s^1_1 s^1_2 s^{1\prime}
%
  \goesto{a^2, \val[\tilde]{2}{}}
%
  s^2_1 s^2_2 s^2
%
  \goesto{a^3, \val[\tilde]{3}{}}
  \cdots
  \goesto{a^k, \val[\tilde]{k}{}}
%
  s^k_1 s^k_2 s^k
  \]
%  
  is a path in $\semopen{(A_1 \arcomp A_2)(\cB)}$.
%
  Notice that, if the above assumption, $a^1 \cap P_1 \neq
  \emptyset$, does not hold, then $s^0_1 = s^1_1$, $s^0 = s^1$
  and we can take $s^{1\prime} = s^1$ in this latter path.

  Repeating the entire argument $k-1$ times, starting with this
  shorter path, we obtain a path
%
  \begin{equation}
    \label{eq:path-in-1}
    s^0_1 s^0
    %
    \goesto{a^1 \cap P_1, \val[\doubletilde]{1}{}}
    %
    s^1_1 s^{1\prime}
    %
    \goesto{a^2 \cap P_1, \val[\doubletilde]{2}{}}
    %
    s^2_1 s^{2\prime}
    %
    \goesto{a^3 \cap P_1, \val[\doubletilde]{3}{}}
    \cdots
    \goesto{a^k \cap P_1, \val[\doubletilde]{k}{}}
    %
    s^k_1 s^{k\prime}
  \end{equation}
%  
  with $s^i \order s^{i\prime}$, for all $i \in [1,k]$.

  From \eq{path-in-1}, we conclude that the state $s^k_1
  s^{k\prime}$ is reachable in $\semopen{A_1(\cB)}$.  Since $A_1$
  enforces $\Phi_1$ on $\cB$, this implies that $s^{k\prime}
  \models \Phi_1$.  Since $s^k \order s^{k\prime}$, we deduce, by
  \defn{property}, that $s^k \models \Phi_1$.

  Symmetrically, $s^k \models \Phi_2$, which concludes
  the proof.
\end{proof}


\end{document}
