% This is LLNCS.DOC the documentation file of
% the LaTeX2e class from Springer-Verlag
% for Lecture Notes in Computer Science, version 2.4
\documentclass{llncs}
\usepackage{llncsdoc}

%\usepackage{a4wide}

\usepackage{graphicx}
\usepackage{verbatim}


\let\proof\relax   
\let\endproof\relax

\usepackage{amsmath,amsthm,amscd}
\usepackage{amssymb}  %for blackboard B
\usepackage{rotating} %to rotate the figures
\usepackage{mathpartir} % math paragraph + inference rules
\usepackage{caption}
\usepackage{xifthen}
\usepackage{stmaryrd} % double brackets (\llbracket, \rrbracket)

%% \usepackage[justification=centering,belowskip=-10pt,aboveskip=0pt]{caption}
%% \setlength{\intextsep}{10pt plus 2pt minus 2pt}
%% \setlength\abovedisplayskip{0pt}

\usepackage{wrapfig}
\usepackage{subcaption}
\usepackage{hyperref}

\usepackage{soul}
\usepackage[colorinlistoftodos,bordercolor=white]{todonotes}
%% \usepackage[disable,colorinlistoftodos,bordercolor=white]{todonotes}
\usepackage{macrospNets}

\newcommand{\Simon}{\\\hfill\mdash Simon}

\newcommand{\noteSB}[2][color=green!40, size=\tiny]{\todo[#1]{{#2}\Simon}}
\newcommand{\noteSBin}[2][inline,color=green!40]{\todo[#1]{{#2}\Simon}}
\newcommand{\todoSB}[2][color=green!40, size=\tiny]{\todo[#1]{\textbf{To-do Simon:} {#2}}}
\newcommand{\todoSBin}[2][inline,color=green!40]{\todo[#1]{\textbf{To-do Simon: } {#2}}}
\newcommand{\Ludo}{\\\hfill\mdash Ludo}
\newcommand{\noteLH}[2][color=green!40, size=\tiny]{\todo[#1]{{#2}\Ludo}}
\newcommand{\noteLHin}[2][inline,color=green!40]{\todo[#1]{{#2}\Ludo}}
\newcommand{\todoLH}[2][color=green!40, size=\tiny]{\todo[#1]{\textbf{To-do Ludo:} {#2}}}
\newcommand{\todoLHin}[2][inline,color=green!40]{\todo[#1]{\textbf{To-do Ludo: } {#2}}}

\newcommand{\defn}[1]{Def.~\ref{defn:#1}}
\newcommand{\defntwo}[2]{Defs~\ref{defn:#1} and \ref{defn:#2}}
\newcommand{\fig}[1]{Fig.~\ref{fig:#1}}
\newcommand{\figs}[2]{Fig.~\ref{fig:#1} and~\ref{fig:#2}}
\newcommand{\tab}[1]{Tab.~\ref{tab:#1}}
\newcommand{\eq}[1]{(\ref{eq:#1})}
\newcommand{\res}[1]{(\ref{res:#1})}
\newcommand{\ex}[1]{Ex.~\ref{ex:#1}}
\newcommand{\exs}[2]{Ex.~\ref{ex:#1} and~\ref{ex:#2}}
\newcommand{\secn}[1]{Sect.~\ref{secn:#1}}
\newcommand{\rem}[1]{Rem.~\ref{rem:#1}}
\newcommand{\lem}[1]{Lem.~\ref{lem:#1}}
\newcommand{\cor}[1]{Cor.~\ref{cor:#1}}
\newcommand{\thm}[1]{Th.~\ref{thm:#1}}
\newcommand{\axs}[1]{Ax.~\ref{ax:#1}}
% \newcommand{\ax}[2]{\ref{ax:#1}.\ref{ax:#1:#2}}
\newcommand{\ax}[1]{Ax.~\ref{ax:#1}}
\newcommand{\prop}[1]{Prop.~\ref{prop:#1}}
\newcommand{\alg}[1]{Alg.~\ref{alg:#1}}
\newcommand{\asmp}[1]{Ass.~\ref{asmp:#1}}

% %%%%%%%%%%%%%%%%%%%%%

\newcommand{\cA}{\ensuremath{\mathcal{A}}}
\newcommand{\bB}{\ensuremath{\mathbf{B}}}
\newcommand{\cB}{\ensuremath{\mathcal{B}}}
\newcommand{\sB}{\ensuremath{\mathbb{B}}}
\newcommand{\cC}{\ensuremath{\mathcal{C}}}
\newcommand{\sC}{\ensuremath{\mathbb{C}}}
\newcommand{\cD}{\ensuremath{\mathcal{D}}}
\newcommand{\fD}{\ensuremath{\mathsf{D}}}
\newcommand{\sD}{\ensuremath{\mathbb{D}}}
\newcommand{\cE}{\ensuremath{\mathcal{E}}}
\newcommand{\sE}{\ensuremath{\mathbb{E}}}
\newcommand{\cF}{\ensuremath{\mathcal{F}}}
\newcommand{\cG}{\ensuremath{\mathcal{G}}}
\newcommand{\cH}{\ensuremath{\mathcal{H}}}
\newcommand{\cI}{\ensuremath{\mathcal{I}}}
\newcommand{\sI}{\ensuremath{\mathbb{I}}}
\newcommand{\cM}{\ensuremath{\mathcal{M}}}
\newcommand{\cN}{\ensuremath{\mathcal{N}}}
\newcommand{\sN}{\ensuremath{\mathbb{N}}}
\newcommand{\cP}{\ensuremath{\mathcal{P}}}
\newcommand{\sP}{\ensuremath{\mathbb{P}}}
\newcommand{\cQ}{\ensuremath{\mathcal{Q}}}
\newcommand{\sQ}{\ensuremath{\mathbb{Q}}}
\newcommand{\cR}{\ensuremath{\mathcal{R}}}
\newcommand{\sR}{\ensuremath{\mathbb{R}}}
\newcommand{\cS}{\ensuremath{\mathcal{S}}}
\newcommand{\sS}{\ensuremath{\mathfrak{S}}}
\newcommand{\cT}{\ensuremath{\mathcal{T}}}
\newcommand{\fT}{\ensuremath{\mathsf{T}}}
\newcommand{\sT}{\ensuremath{\mathbb{T}}}
\newcommand{\cV}{\ensuremath{\mathcal{V}}}
\newcommand{\fV}{\ensuremath{\mathsf{V}}}
\newcommand{\sV}{\ensuremath{\mathbb{V}}}
\newcommand{\sZ}{\ensuremath{\mathbb{Z}}}

\newcommand{\mdash}[1][]{---#1}
\newcommand{\ndash}{--}
\newcommand{\ie}[1][\ ]{i.e.#1}
\newcommand{\etc}[1][\ ]{etc.#1}
\newcommand{\eg}[1][\ ]{e.g.#1}
\newcommand{\cf}[1][\ ]{cf.#1}
\newcommand{\wrt}[1][\ ]{w.r.t.#1}

\newcommand{\bydef}[1]{\ensuremath{\stackrel{\mathit{\scriptscriptstyle def}}{#1}}}
\newcommand{\suchthat}{\ensuremath{\,|\,}}
\newcommand{\rightsuchthat}{\ensuremath{\,\right|\,}}
\newcommand{\leftsuchthat}{\ensuremath{\,\left|\,}}
\newcommand{\setdef}[2]{\ensuremath{\{{#1}\,|\,{#2}\}}}
\newcommand{\setdefb}[2]{\ensuremath{\bigl\{{#1}\,\bigl|\,{#2}\bigr.\bigr\}}}
\newcommand{\Setdef}[2]{\ensuremath{\Big\{{#1}\,\Big|\,{#2}\Big\}}}
\newcommand{\goesto}[2][]{\ensuremath{\xrightarrow[#1]{#2}}}
\newcommand{\notgoesto}[2][]{\ensuremath{\not\xrightarrow[#1]{\ \,#2}}}
\newcommand{\non}[1]{\ensuremath{\overline{#1}}}

\newcommand{\true} {\ensuremath{\mathtt{t\!t}}}
\newcommand{\false}{\ensuremath{\mathtt{f\!f}}}
\newcommand{\noop} {\ensuremath{\mathsf{skip}}}

\newcommand{\data}{\ensuremath{\sD}}
\newcommand{\guards}[1]{\ensuremath{\sB_{#1}}}
\newcommand{\exprs}[1]{\ensuremath{\sE_{#1}}}
\newcommand{\valuations}[1]{\ensuremath{\sV_{#1}}}
\newcommand{\val}[3][]{%
  \ensuremath{\sigma^{#2}_{#3}%
    \ifthenelse{\isempty{#1}}{}{(#1)}%
  }%
}
%% \DeclareMathOperator{\supp}{supp}
\newcommand{\supp}[1]{\ensuremath{\mathrm{supp}(#1)}}
\newcommand{\semopen}[1]{\ensuremath{[{#1}]}}
\newcommand{\semclosed}[1]{\ensuremath{\llbracket{#1}\rrbracket}}
\newcommand{\reachable}[1]{\ensuremath{\mathit{reachable}({#1})}}
\newcommand{\IMextend}[2]{\ensuremath{#1 \ltimes #2}}

% %%%%%%%%%%%%%%%%%%%%%

\usepackage{tikz}
\usetikzlibrary{calc}
\usetikzlibrary{arrows,shapes,automata,petri}
  \tikzset{
  place/.style={
    circle,
    thick,
    draw=blue!75,
    fill=blue!20,
    minimum size=6mm
  },
  transition/.style={
    rectangle,
    thick,
    fill=black,
    minimum width=8mm,
    inner ysep=2pt
    },
    arc/.style = {
      decoration=
      {markings,mark=at position #1 with {circle;}
      },
      postaction={decorate,draw}},
  }   

\let\llncssubparagraph\subparagraph
\let\subparagraph\llncssubparagraph

\begin{document}
\graphicspath{{figures/}}

\title{Architectures and Open pNets}

\author{%
Simon~Bliudze\inst{1}
\and
Ludovic~Henrio\inst{2}
\and
Eric~Madelaine\inst{3}
\and
...
}

\institute{%
  INRIA Lille -- Nord Europe, Villeneuve d'Ascq, France\\
  \email{simon.bliudze@inria.fr}
\and
\and
}


\maketitle

\begin{abstract}
  We extend the theory of architectures with a general
  mechanism for handling various types of data transfer.  We
  encode this extended model into the open pNet semantic
  model and show how this encoding can be used to verify
  that an architecture does, indeed, enforce its
  characteristic property.

\keywords{}
\end{abstract}

%****************************************************************
%****************************************************************

\section{Introduction}
\label{secn:introduction}

%****************************************************************
%****************************************************************

\section{The theory of architectures with data}
\label{secn:archs}

%****************************************************************

\subsection{Components and composition}
\label{sec:components}

\todoSBin{Preliminaries on the universal(?) data domain $\data$
  and valuations.}

\begin{definition}[Components]
  \label{defn:component}
  A \emph{component} is a tuple $(Q, q^0, P, V, \val{0}{},
  \goesto{})$, where
  \begin{itemize}
  \item $Q$ is a set of \emph{states}, with $q^0 \in Q$ the
    \emph{initial state}, 
  \item $P$ is a set of \emph{ports},
  \item $V$ is a set of \emph{component variables},
  \item $\val{0}{} : V \rightarrow \data$ is an \emph{initial
    valuation} of the component variables, 
  \item $\goesto{}\, \subseteq
    Q \times 2^P \times \guards{V} \times \exprs{V} \times Q$
%
    is a \emph{transition relation}, with transitions
    labelled by triples consisting of an \emph{interaction}
    $a \subseteq P$, a Boolean \emph{guard} $g \in
    \guards{V}$ and an \emph{update expression} $e \in
    \exprs{V}$.
  \end{itemize}
%
  We call the pair of sets $(P,V)$ the \emph{interface} of the
  component.\,\footnote{%
%
  In practice, component variables are split in two sub-sets:
  local and exported variables.  Only exported variables belong
  to the component interface and can be used for the interactions
  with other components (see \defn{im}).  However, in the context
  of this paper we can simplify by omitting this separation.
%
  }
%
  Notations $q \goesto{a} q'$, $q \goesto{a}$ and $q
  \notgoesto{a}$ are as usual; for a component $B$, we denote
  $Q_B$, $q^0_B$, $P_B$, $V_B$, $\val{0}{B}$ and $\goesto[B]{}$
  the constituents of $B$.
\end{definition}

\begin{definition}[Component semantics]
  \label{defn:comp:semantics}
  The \emph{open semantics} of a component $B = (Q, q^0, P, V,
  \val{0}{}, \goesto{})$ is given by the LTS $\semopen{B} = (S,
  s^0, \goesto{})$, where $S = Q \times \valuations{V}$, $s^0 =
  (q^0, \val{0}{})$ and $\goesto{}$ is the minimal transition
  relation satisfying the following rule:
  %
  \begin{equation}
    \label{eq:comp:semantics}
    \infer{
      q \goesto{a, g, e} q'
      \and
      v \models g
      \and
      v' = v[e]
    }{
      (q, v) \goesto{a} (q', v')
    }
    \,.
  \end{equation}
  The \emph{closed semantics} of $B$ is given by the LTS
  $\semclosed{B} \bydef{=}\reachable{\semopen{B}}$, comprising
  only the reachable states of $\semopen{B}$.
\end{definition}

\begin{definition}[Interaction model \& composition]
  \label{defn:im}
  Let $\cB = \{B_1,\dots,B_n\}$ be a finite set of
  components with $B_i = (Q_i, q^0_i, P_i, V_i, \val{0}{i},
  \goesto{})$,\footnote{%
%
    Here and below, we skip the index on the transition
    relation $\goesto{}$, since it is always clear from the
    context.
%
} such that all their respective components (\ie $Q_i$, $P_i$,
  and $V_i$) are pairwise disjoint, \ie $\forall i \neq j,\ Q_i
  \cap Q_j = P_i \cap P_j = V_i \cap V_j = \emptyset$.  Let $P =
  \bigcup_{i = 1}^n P_i$ and $V = \bigcup_{i = 1}^n V_i$.

  An \emph{interaction model over $(P,V)$} is a set $\Gamma
  \subseteq 2^P \times \guards{V} \times \exprs{V}$, such that,
  for any $(a, g, e) \in \Gamma$, the guard and update expression
  associated to the \emph{interaction} $a$ satisfy, respectively,
  $g \in \guards{V_a}$ and $e \in \exprs{V_a}$, with $V_a
  \bydef{=} \bigcup_{i:\, a \cap P_i \neq \emptyset} V_i$.  We
  denote $\supp{a} \bydef{=} \setdef{i \in [1,n]}{a \cap P_i \neq
    \emptyset}$ the \emph{support} of the interaction $a$.
\noteLH{why not introduce supp earlier and use it for $V_a$?}
  %% We call the set of ports $P$ the \emph{domain} of the
  %% interaction model.

  The \emph{composition of $\cB$ with the interaction model
    $\Gamma$} is the component $\Gamma(\cB) = (Q, q^0, P, V,
  \val{0}, \goesto{})$, where $Q = \prod_{i=1}^n Q_i$, $q^0 =
  q_1^0\dots q_n^0$, $\val{0}{}: V \rightarrow \data$ is such
  that, for any $v \in V_i$, $\val[v]{0}{} = \val[v]{0}{i}$, and
  $\goesto{}$ is the minimal transition relation satisfying
  following rules:
%
  \begin{mathpar}
    \infer{
      q_i \goesto{\emptyset, \true, \noop} q_i'
    }{
      q_1 \dots q_i \dots q_n \goesto{\emptyset, \true, \noop} q_1 \dots q_i' \dots q_n
    }\,,
    
    \infer{    
      (a, g, e) \in \Gamma
      \and
      \forall i \in \supp{a}, q_i \goesto{a \cap P_i, g_i, e_i} q_i'
      \and
      \forall i \not\in \supp{a}, q_i = q_i'
      \\\\
      G = g \land \bigwedge_{i \in \supp{a}} g_i
      \and
      E = e; (e_i)_{i \in \supp{a}}
    }{
      q_1\dots q_n \goesto{a, G, E} q_1'\dots q_n'
    }\,.
  \end{mathpar}
\end{definition}

Below, when speaking of a set of components $\cB$, we will always
assume that it satisfies all the assumptions of \defn{im}.
%
For an interaction $a$, we will abuse the notation by writing
$\supp{a}$ to also denote the set $\setdef{B \in \cB}{a \cap P_B
  \neq \emptyset}$.  The precise meaning of this notation will
always be clear from the context.

%****************************************************************
\subsection{Architectures}
\label{secn:archi}

\begin{definition}[Architecture]
  \label{defn:arch}
  An \emph{architecture} is a tuple $A = (\cC, P_A, V_A, \Gamma)$,
  where $\cC$ is a finite set of \emph{coordinating components}
  with pairwise disjoint sets of ports and variables, such that
  $\bigcup_{C \in \cC} P_C \subseteq P_A$ and
  $\bigcup_{C \in \cC} V_C \subseteq V_A$, and
  $\Gamma \subseteq 2^{P_A} \times \guards{V_A} \times \exprs{V_A}$
  is an interaction model over $(P_A, V_A)$.
\end{definition}

\begin{definition}[Application of an architecture]
  \label{defn:arch:application}
  Let $A = (\cC, P_A, V_A, \Gamma)$ be an architecture and let $\cB$
  be a set of components, such that
%
  \begin{align}
    \bigcup_{B \in \cB} P_B \cap \bigcup_{C \in \cC} P_C = \emptyset\,,
    &&
    P_A \subseteq P \bydef{=} \bigcup_{B \in \cB \cup \cC} P_B\,,
    \\
    \bigcup_{B \in \cB} V_B \cap \bigcup_{C \in \cC} V_C = \emptyset\,,
    &&
    V_A \subseteq V \bydef{=} \bigcup_{B \in \cB \cup \cC} V_B\,,
  \end{align}
%
  and, denoting, for any $(a, g, e) \in \Gamma$, $V_a \bydef{=}
  \bigcup_{B \in \supp{a}} V_B$, we have $g \in \guards{V_a}$ and
  $e \in \exprs{V_a}$.
%
  The \emph{application of the architecture $A$ to the set of
  components $\cB$} is the component $ A(\cB) \bydef{=}
  (\IMextend{\Gamma}{P})(\cC \cup \cB)$, where
%
  \begin{equation}
    \label{eq:im:extension}
    \IMextend{\Gamma}{P} \bydef{=}
    \setdefb{
      (a \cup a', g, e)
    }{
      (a, g, e) \in \Gamma, a' \subseteq P \setminus P_A
    }
  \end{equation}
%
  denotes the \emph{extension} of the interaction model $\Gamma$
  to the set of ports $P$.
\end{definition}

An architecture $A$ enforces coordination constraints on the
components in $\cB$.  The interface $(P_A, V_A)$ of an
architecture $A$ contains all ports of the coordinating
components $\cC$ and some additional ports, which must belong to
the components in $\cB$.  In the application $A(\cB)$, the ports
belonging to $P_A$ can only participate in the interactions
defined by the interaction model $\Gamma$ of $A$.  Ports which do
not belong to $P_A$ are not restricted and can participate in any
interaction.  In particular, they can join the interactions in
$\Gamma$ (see \eq{im:extension}).  If the interface of the
architecture covers all ports of the system, \ie $P = P_A$, we
have $P\setminus P_A = \emptyset$ and the only interactions
allowed in $A(\cB)$ are those belonging to $\Gamma$.  Notice also
that the restrictions imposed by \defn{arch:application} on the
set of operand components $\cB$ ensure that an interaction in
$\IMextend{\Gamma}{P}$ can only refer to variables of the
participating components.
%
Finally, the definition of $\IMextend{\Gamma}{P}$ requires that
an interaction from $\Gamma$ be involved in every interaction
belonging to $\IMextend{\Gamma}{P}$.  To allow the ports from $P
\setminus P_A$ to be fired independently in $A(\cB)$, one must
have $(\emptyset, \true, \noop) \in \Gamma$.  

\begin{definition}[Composition of architectures]
  \label{defn:arch:composition}
  
\end{definition}

%****************************************************************
%****************************************************************

\section{Open pNets}
\label{secn:pNets}

pNets are tree-like structures, where the leaves are either
\emph{parameterised labelled transition systems (pLTSs)}, expressing the
behaviour of basic processes, or \emph{holes}, used as placeholders
for unknown processes, of which we only specify the set of possible
actions, this set is named the \emph{sort}.
Nodes of the tree (pNet nodes) are synchronising artifacts, using a
set of \emph{synchronisation vectors} that express the possible
synchronisation between the parameterised actions of a subset of the
sub-trees.


%\smallskip\noindent
\paragraph*{Notations.}
We extensively use indexed structures
over some countable indexed sets, which are equivalent to mappings over
the countable set. % . The indexes will usually be
% integers, bounded or not. Such an indexed family is
%denoted
%follows:
$a_i^{i\in I}$
%, or equivalently  $(i\mapsto a_i)^{i\in I}$
denotes a family of elements $a_i$ indexed over the
set $I$. % Such a family
% is equivalent to the mapping $(i\mapsto a_i)^{i\in I}$.
% To specify the set over which the structure is indexed,
% indexed structures are always denoted with an exponent of the form $i\in I$
% (arithmetic only appears in the indexes if necessary).
$a_i^{i\in I}$ defines both $I$ the set over which the family is
indexed (called \emph{range}), and $a_i$ the elements of the family.
E.g., $a^{i\in\{3\}}$ is the mapping with a single entry $a$ at index
$3$ ; abbreviated $(3\mapsto a)$ in the following.
When this is not
ambiguous, we shall use notations for sets, and typically write
``indexed set over I'' when formally we should speak of multisets, and
write $x\in a_i^{i\in I}$ to mean $\exists i\in I.\, x=a_i$.  An empty
family is denoted $\emptyset$. We
denote classically with an overline -- $\overline{a}$  -- a family when the indexing set 
is
not meaningful.  $\uplus$ is the disjoint union on
indexed sets.

\paragraph*{Term algebra.}
Our models rely on a notion of parameterised actions, that are
symbolic expressions using data types and variables. As our model aims
at encoding the low-level behaviour of possibly very different
programming languages, we do not want to impose one specific algebra
for denoting actions, nor any specific communication mechanism. So we
leave unspecified the constructors of the algebra that will allow building
expressions and actions. Moreover, we use a generic {\em action interaction}
mechanism, based on (some sort of) unification between two or more action
expressions, to express various kinds of communication or
synchronisation mechanisms.

\def\Talg{\mathcal{T}_{\Sigma,\P}}
Formally, we assume the existence of a term algebra $\Talg$,
where $\Sigma$ is the signature of the data and action constructors,
and $\P$ a set of variables. Within $\Talg$, we distinguish a set of
data expressions $\mathcal{E}_\P$, including a set of boolean
expressions $\mathcal{B}_{\P}$ ($\mathcal{B}_{\P}\subseteq\mathcal{E}_\P$).
On top of $\mathcal{E}_\P$ we build the action algebra
$\mathcal{A}_\P$, with $\mathcal{A}_P\subseteq\mathcal{T}_\P,
\mathcal{E}_P\cap\mathcal{A}_P=\emptyset$;
naturally action terms will use data expressions as subterms.
To be able to reason about the data flow between pLTSs, we
distinguish \emph{input variables} of the form $?x$ within terms; the function
$\vars(t)$ identifies the set of variables in a term
$t\in\AlgT$, and $iv(t)$ returns its input variables.


pNets can encode naturally the notion of input actions in value-passing CCS
\cite{Milner89} or of usual point-to-point message passing calculi, but it also allows
for more general mechanisms, like gate negociation in Lotos, or broadcast
communications. Using our notations, value-passing actions \emph{\`a la} CCS would be
encoded as $a(?x_1,...,?x_n)$ for inputs, $a(v_1,..,v_n)$ for outputs (in which $v_i$ are 
action terms containing no input variables). 
%For Lotos-style
%distributed synchronisation, we use synchronisation vectors that encode matching data
%offers on a common gate; or 
We can also use more complex action structure such as Meije-SCCS action
monoids, like in $a.b$, $a^{f(n)}$ (see \cite{deSimone85}). The expressiveness of the 
synchronisation constructs
will depend on the action algebra.




\subsection{The (open) pNets Core Model}
\label{section:pNets}


A pLTS is a labelled transition system with variables; variables can be
manipulated, defined, or accessed inside states, actions, guards, and
assignments. Without loss of generality and to simplify the formalisation, we suppose 
here that 
variables are local to each 
state: each state has its set of variables disjoint from the others. Transmitting 
variable values from one state to the other can be done by explicit assignment. 
%Similarly, to simplify the management of variables and without loss of expressivity, we 
%suppose that transitions looping to the same state does not do assignments.
Note that we make no assumption on finiteness of the set of states nor
on finite branching of the transition relation.

We first define the set of actions a pLTS can use, let $a$
range over action labels, $\symb{op}$ are operators, and $x_i$ range over
variable names. Action terms are:
\[
\begin{array}[l]{rcl@{\quad}p{5.5cm}}
  \alpha\in\AlgA&::=&a(p_1,\ldots,p_n)&\text{action terms}\\
  p_i&::=& ?x~|~\symb{Expr}&\text{parameters (input variable or expression)}\\
  \symb{Expr}&::=& 
  \symb{Value}~|~x~|~\symb{op}(\symb{Expr}_1,..,\symb{Expr}_n)&\text{Expressions}
\end{array}
\]
The input variables in an action term are those marked with a
$\symb{?}$.
We additionally suppose that each input variable does not
appear somewhere else in the same action term:
$p_i=?x\Rightarrow\forall j\neq i.\, x\notin \vars(p_j)$

\begin{definition}[pLTS]
\label{pLTS}
A pLTS is a tuple
$\pLTS\triangleq\mylangle S,s_0, \to\myrangle$ where:
\begin{itemize}
\item[$\bullet$]
$S$ is a set of states.
\item[$\bullet$]
$s_0 \in S$ is the initial state.
%\item[$\bullet$]
 %Variables in
%$\iv(\alpha)$ are assigned by the action, other variables can be assigned
%by the additional assignments.
\item[$\bullet$] $\to \subseteq S \times L \times S$ is the transition relation and 
$L$ is the set of labels of the form
$\langle \alpha,~e_b,~(x_j\!:= {e}_j)^{j\in J}\rangle$,
where $\alpha \in\AlgA$ is a parameterised action, $e_b \in
\AlgB$ is a guard, and the variables $x_j\in P$
are assigned the expressions $e_j\in \AlgE$.
If 
$s \xrightarrow{\langle \alpha,~e_b,~(x_j\!:= {e}_j)^{j\in
		J}\rangle} s'\in \to $ then $\iv(\alpha)\!\subseteq\! \vars(s')$, 
		$\vars(\alpha)\backslash \iv(\alpha)\!\subseteq\! \vars(s)$, 
		$\vars(e_b)\!\subseteq\! \vars(s')$, and
		$\forall j\!\in\! J .\,\vars(e_j)\!\subseteq\! \vars(s)\land 
		x_j\!\in\!\vars(s')$. %,  and $s= s'\Rightarrow J=\emptyset$. 
		
\end{itemize}
\end{definition}

Now we define
pNet nodes, as constructors for hierarchical behavioural structures.
A pNet has a set of sub-pNets that can be either pNets or pLTSs, and a
set of Holes, playing the role of process parameters.

A composite pNet consists of a set of sub-pNets exposing
a set of actions, each of them triggering internal actions in each of
the sub-pNets. The synchronisation between global actions and
internal actions is given by  \emph{synchronisation vectors}: a
synchronisation vector synchronises one or several internal actions, and
exposes a single resulting global action.
Actions involved at the pNet level (in the synchronisation vectors) do
not need to distinguish between input and output
variables. Action terms for pNets are defined as follows:
\[\begin{array}[l]{rcl@{\quad}l}
  \alpha\in \AlgAS &::=&a(Expr_1,\ldots,Expr_n)
\end{array}
\]



\begin{definition}[pNets]\label{def-pnets}
A pNet is a hierarchical structure where leaves are pLTSs and holes:\\
$\pNet\triangleq \pLTS~|~\mylangle \pNet_i^{i\in I}, \Sort_j^{j\in J}, \symb{SV}_k^{k\in 
K}\myrangle$
where
\begin{itemize}
\item[$\bullet$] $I \in \I$ is the set over which sub-pNets are indexed.
\item[$\bullet$] $\pNet_i^{i\in I}$ is the family of sub-pNets.
%  $\pNet_i^{i\in I}$ is a family of sub-pNets where $I\in\I_\P$ is the set over which 
%sub-pNets are indexed.

\item[$\bullet$] $J\!\in\!\I_\P$ is a set of indexes, called \emph{holes}.
$I$ and $J$ are \emph{disjoint}: $I\!\cap\! J=\emptyset$,  $I\!\cup\! J\neq\emptyset$
\item[$\bullet$] $\Sort_j \subseteq \AlgAS$ is a set of action terms, denoting the 
\emph{sort} of
hole $j$.

\item[$\bullet$] $\symb{SV}_k^{k\in K}$ is a set of
  synchronisation vectors ($K\in\I_\P$). $\forall k\!\in\! K,
  \symb{SV}_k\!=\!\alpha_{l}^{l\in I_k \uplus J_k}\to\alpha'_k$ where
  $\alpha'_k\in \mathcal{A}_\P$, $I_k\subseteq I$, $J_k\subseteq J$,
  $\forall i\!\in\!
  I_k.\,\alpha_{i}\!\in\!\Sort(\pNet_i)$,  $\forall j\!\in\!
  J_k.\,\alpha_{j}\!\in\!\Sort_j$, and $\vars(\alpha'_k)\subseteq \bigcup_{l\in I_k\uplus 
  J_k}{\vars({\alpha_l})}$. The global action of a vector $\symb{SV}_k$ is
$\Label(\symb{SV}_k) = \alpha'_k$.


\end{itemize}
\end{definition}

The preceding definition relies on the auxiliary functions below:

\begin{definition}[Sorts, Holes, Leaves of pNets]
  \begin{itemize}
  \item The sort of a pNet is its signature, i.e. the set of actions it can
perform. In the definition of sorts, we do not need to distinguish
input variables (that specify the dataflow within LTSs), so for
computing LTS sorts, we use a substitution operator\footnote{$\subst{y_k\gets x_k}^{k\in 
K}$ is the parallel substitution 
operation.} to remove the
\emph{input marker} of variables. Formally:
\[
\begin{array}{l}
\Sortop(\mylangle S,s_0, \to\myrangle) = \{\alpha\subst{x \gets ?x| 
x\in\symb{iv}(\alpha)}|s \xrightarrow{\langle \alpha,~e_b,~(x_j\!:= {e}_j)^{j\in
    J}\rangle} s'\in \to \} \\
\Sortop(\mylangle \overline{\pNet}\!, %\pNet_i^{i\in I}, \Sort_j^{j\in J}
\overline{\pNet}\!,
\overline{\symb{SV}}\myrangle)
=\{\alpha' |\, \overline{\alpha}%\alpha_j^{j\in J_k}
\to\alpha'\in\set{\symb{SV}}\}
\end{array}
\]

\item
The set of holes of a pNet is defined inductively; the sets of holes
in a pNet node and its subnets are all disjoint:
  \[\begin{array}{l}
\Holes(\mylangle S,s_0, \to\myrangle) \!=\! \emptyset \\
\Holes(\mylangle \pNet_i^{i\in I}\!,\Sort_j^{j\in J}\!, \overline{\symb{SV}}\myrangle) 
=J\uplus{\displaystyle \bigcup_{i\in 
I}\Holes(\pNet_i)}\\
\forall i\in I.\, \Holes(\pNet_i)\cap J=\emptyset\\
\forall i_1,i_2\in I.\,i_1\neq i_2\Rightarrow  
\Holes(\pNet_{i_1})\cap\Holes(\pNet_{i_2})=\emptyset
\end{array}\]
\item
The set of leaves of a pNet is the set of all pLTSs occurring in the structure, defined 
inductively as:
\noteLH{Not very nice but at least of the right kind for the definition of composition}
\[\begin{array}{l}
\Leaves(\mylangle S,s_0, \to\myrangle) \!=\!\emptyset\\
\Leaves(\mylangle \pNet_i^{i\in I}\!,%Sort_j^{j\in J}
\overline{\Sort}\!, \overline{\symb{SV}}\myrangle) = {\displaystyle \biguplus_{i\in 
I}\Leaves(\pNet_i)\uplus\{i\mapsto \pNet_i|\pNet_i=\mylangle S,s_0, \to\myrangle\}}
\end{array}\]
\end{itemize}
\end{definition}



A pNet $Q$ is \emph{closed} if it has no hole: $\Holes(Q)=\emptyset$; else it
is said to be \emph{open}.



%****************************************************************
%****************************************************************

\section{Encoding of architectures into open pNets}
\label{secn:encoding}

%****************************************************************
%****************************************************************

\section{SMT encoding of open pNets}
\label{secn:smt}

%****************************************************************
%****************************************************************

\section{Case study}
\label{secn:case-study}

%****************************************************************
%****************************************************************

\section{Related work}
\label{secn:related}

%****************************************************************
%****************************************************************

\section{Conclusion}
\label{secn:conclusion}

%****************************************************************
%****************************************************************

\bibliographystyle{abbrv}
\bibliography{biblio.bib}

\end{document}
